% Filename:   ass01    
% Version:    1.0
% Since:      2013-03-15
% Author: 
%    Jimmy Lin (u5223173) 
%    - u5223173@uds.anu.edu.au
%    - linxin@gmail.com        
%     
% Edited by MacVim
% Documentation auto-generated by Snippet 

\documentclass{article}
% Change "article" to "report" to get rid of page number on title page
\usepackage{amsmath,amsfonts,amsthm,amssymb}
\usepackage{setspace}
\usepackage{Tabbing}
\usepackage{fancyhdr}
\usepackage{lastpage}
\usepackage{extramarks}
\usepackage{chngpage}
\usepackage{soul,color}
\usepackage{graphicx,float,wrapfig}

\usepackage{tikz}
\usetikzlibrary{shapes,arrows}

% In case you need to adjust margins:
\topmargin=-0.45in      %
\evensidemargin=0in     %
\oddsidemargin=0in      %
\textwidth=6.5in        %
\textheight=9.0in       %
\headsep=0.25in         %

\newcommand{\hs}[1]{\hspace{#1}}

% Assignment Information
\newcommand{\CourseCode}{COMP3630}
\newcommand{\CourseName}{Theory of Computation}
\newcommand{\TaskTitle}{Assignment 01}
\newcommand{\OutDate}{08 March 2013}
\newcommand{\DueDate}{22 March 2013}
% Default Information
\newcommand{\Major}{Advanced Computing (R\&D)}
\newcommand{\University}{Australian National University}
\newcommand{\StudentName}{Jimmy Lin}
\newcommand{\UID}{u5223173}
\newcommand{\Email}{linxin@gmail.com}


% Define block styles
\tikzstyle{line} = [draw, -latex']
\tikzstyle{circle} = [draw, ellipse,fill=blue!20, node distance=3cm,
    minimum height=2em]

%%%%%%%%%%%%%%%%%%%%%%%%%%%%%%%%%%%%%%%%%
\begin{document}
% header
\begin{center}
    \LARGE{\textbf{\CourseCode: \CourseName} } \\
    \large{\textbf{\TaskTitle \hs{2mm} Due: \DueDate}} \\
    \normalsize{UID: \UID \hs{2mm} Name: \StudentName \hs{2mm} E-mail: \Email} \\
    \small{\Major \hs{3mm} \University}  \\
\end{center}
\hrule .\\
% Exercise 1
\textbf{Exercise 1 DFAs and NFAs\\}
\textbf{1. DFA\\}
\begin{center}
\begin{tikzpicture}[->,>=stealth',shorten >=1pt,auto,node distance=2.4cm,on grid,semithick,
every state/.style={fill=blue!10,thick}]

\node[state,initial,initial text={}] (q1) {$q_1$};
\node[state] (q2) [right=of q1] {$q_2$};
\node[state,accepting] (q3) [right=of q2] {$q_3$};
\node[state,accepting] (q4) [below=of q3] {$q_4$};
\node[state] (q5) [right=of q3] {$q_5$};
\node[state] (q6) [above right=of q5] {$q_6$};
\node[state] (q7) [below right=of q5] {$q_7$};
\node[state,accepting] (q8) [below right=of q6] {$q_8$};

\draw[every node/.style={font=\footnotesize}]
       (q1) edge node{$B$/R} (q2)
       (q2) edge[loop above] node{1,2/R} ()
       (q2) edge node{$B$/L} (q3)
       (q3) edge node{2/1} (q4)
       (q3) edge node{1/L} (q5)
       (q5) edge[loop above] node{1/L} ()
       (q5) edge[swap] node{2/1} (q6)
       (q5) edge node{$B$/R} (q7)
       (q6) edge[swap] node{1/R} (q8)
       (q7) edge[loop below] node{1/$B$} ()
       (q7) edge node{$B$/R} (q8)
       (q8) edge[loop above] node{1/2} ()
       (q8) edge[loop below] node{2/R} ();
\end{tikzpicture}
\end{center}

\end{document}
%%%%%%%%%%%%%%%%%%%%%%%%%%%%%%%%%%%%%%%%%
