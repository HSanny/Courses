%%%%%%%%%%%%%%%%%%%%%%%%%%%%%%%%%%%%%%%%%%%%%%%%%%%%%%%%%%%%%%%%%%%%%%%%
%%%  THIS TEX FILE IS TO GENERATE PDF FILE FOR 
%%% 
%%%  COPYRIGHT (C) JIMMY LIN, 2013, UT AUSTIN
%%%%%%%%%%%%%%%%%%%%%%%%%%%%%%%%%%%%%%%%%%%%%%%%%%%%%%%%%%%%%%%%%%%%%%%%
\documentclass[11pt,a4paper]{article}
%%%%%%%%%%%%%%%%%%%%%%%%%%%%%%%%%%%%%%%%%%%%%%%%%%%%%%%%%%%%%%%%%%%%%%%%
%%%  PACKAGES USED IN THIS TEX SOURCE FILE
%%%%%%%%%%%%%%%%%%%%%%%%%%%%%%%%%%%%%%%%%%%%%%%%%%%%%%%%%%%%%%%%%%%%%%%%
\usepackage{geometry,amsthm,amsmath,graphicx,fancyheadings,fancybox}
\usepackage{tikz}
\usepackage[colorlinks,
            linkcolor=blue,
            anchorcolor=red,
            citecolor=green
            ]{hyperref}
\usepackage{/Users/JimmyLin/workspace/latexTemplate/UTA_CS/JS}
\usepackage{/Users/JimmyLin/workspace/latexTemplate/UTA_CS/JSASGN}
%%%%%%%%%%%%%%%%%%%%%%%%%%%%%%%%%%%%%%%%%%%%%%%%%%%%%%%%%%%%%%%%%%%%%%%%
%%% MACROS CONTAINING THE FILE INFORMATION
%%%%%%%%%%%%%%%%%%%%%%%%%%%%%%%%%%%%%%%%%%%%%%%%%%%%%%%%%%%%%%%%%%%%%%%%
\renewcommand{\COURSE}{CS331 Algorithm}
\renewcommand{\LECTURER}{Greg Plexton}
\renewcommand{\TUTOR}{53735}
\renewcommand{\TASK}{Assignment 06}
\renewcommand{\RELEASEDATE}{April. 16 2014}
\renewcommand{\DUEDATE}{April. 24 2014}
\renewcommand{\TIMECONSUME}{15 hours}
%%%%%%%%%%%%%%%%%%%%%%%%%%%%%%%%%%%%%%%%%%%%%%%%%%%%%%%%%%%%%%%%%%%%%%%%
%%% DOCUMENTATION STARTS FROM HERE 
%%%%%%%%%%%%%%%%%%%%%%%%%%%%%%%%%%%%%%%%%%%%%%%%%%%%%%%%%%%%%%%%%%%%%%%%
\begin{document}
%%%%%%%%%%%%%%%%%%%%%%%%%%%%%%%%%%%%%%%%%%%%%%%%%%%%%%%%%%%%%%%%%%%%%%%%
%% TITLE PAGE
%%%%%%%%%%%%%%%%%%%%%%%%%%%%%%%%%%%%%%%%%%%%%%%%%%%%%%%%%%%%%%%%%%%%%%%%
\begin{titlepage}
    \maketitle
\end{titlepage}
%%%%%%%%%%%%%%%%%%%%%%%%%%%%%%%%%%%%%%%%%%%%%%%%%%%%%%%%%%%%%%%%%%%%%%%%
%% CONTENT PAGE: TABLEOFCONTENTS, LISTOFTABLES, LIST OF FIGURES
%%%%%%%%%%%%%%%%%%%%%%%%%%%%%%%%%%%%%%%%%%%%%%%%%%%%%%%%%%%%%%%%%%%%%%%%
\renewcommand{\contentsname}{Contents}
\begin{center} 
    \tableofcontents 
\end{center}
\newpage
%%%%%%%%%%%%%%%%%%%%%%%%%%%%%%%%%%%%%%%%%%%%%%%%%%%%%%%%%%%%%%%%%%%%%%%%
%%% GENERAL DOCUMENTATION BEGINS 
%%%%%%%%%%%%%%%%%%%%%%%%%%%%%%%%%%%%%%%%%%%%%%%%%%%%%%%%%%%%%%%%%%%%%%%%
\newcommand{\gapjk}{\ensuremath{gap(p,j,k)}}
\newcommand{\gapij}{\ensuremath{gap(p,i,j)}}
\newcommand{\gapik}{\ensuremath{gap(p,i,k)}}
\newcommand{\gjk}{\ensuremath{g(j,k)}}
\newcommand{\gij}{\ensuremath{g(i,j)}}
\newcommand{\gik}{\ensuremath{g(i,k)}}
\newcommand{\fpij}{\ensuremath{f(p,i,j)}}
\newcommand{\fpjk}{\ensuremath{f(p,j,k)}}
\newcommand{\fpik}{\ensuremath{f(p,i,k)}}
\newcommand{\s}[1]{\ensuremath{s(#1)}}
\newcommand{\al}[1]{\ensuremath{\alpha(#1)}}
\newcommand{\qi}{\ensuremath{q_{\beta(i)}}}
\newcommand{\qj}{\ensuremath{q_{\beta(j)}}}
\newcommand{\qk}{\ensuremath{q_{\beta(k)}}}

\section{Exercise 1}
\ovalbox{
\begin{minipage}{17cm} 
    Let $i$, $j$, and $k$ be distinct integers in $[n]$ such that $j$ lies
    between $i$ and $k$, i.e., either $i < j < k$ and $k < j < i$. Assume that 
    bids $\alpha(i)$ and $\alpha(j)$ are linear. Prove that if $gap(p,i,j) =
    0$ and $\gjk = 0$, then $\gij = 0$.
\end{minipage} } \\[0.5cm]

Based on the known condition as follows, 
\begin{align}
    \gapij = 0 \\
    \gapjk = 0
\end{align}

And the assumed condition that 
\begin{align}
    i,j,k \in [n] \\
    \al{j} \text{ is linear}  \\
    \al{i} \text{ is linear} 
\end{align}

We have 
\begin{align}
    max(0, \s{i} \cdot \gij - \fpij) = 0 \\
    max(0, \s{j} \cdot \gjk - \fpjk) = 0
\end{align}

That is, 
\begin{align}
    \s{i} \cdot \gij - \fpij \leq 0 \label{1.1.A} \\
    \s{j} \cdot \gjk - \fpjk \leq 0 \label{1.1.B}
\end{align}

We sum up \eqref{1.1.A} and \eqref{1.1.B}, then get
\begin{align}
    \s{i} \cdot \gij - \fpij + \s{j} \cdot \gjk - \fpjk \leq 0 
\end{align}

This can be simplified to be 

\begin{align}
    \s{i} \cdot \gik - \fpik \leq  \gjk \cdot (\s{i} - \s{j})  
\end{align}

According to the known condition, we have
\begin{align}
    i < j < k \ \text{or }  k < j < i 
\end{align}

In the case of $i < j < k$, we have $\gjk = \qk - \qj \geq 0$ and 
$\s{i} - \s{j} \leq 0$ since $\al{i}$ and $\al{j}$ are matched to $\beta{i}$ and
$\beta{j}$ in MWMCM $M$, respectively. Similarly, in the case of $k < j < i$,
we have $\gij = \qj - \qi \leq 0$ and $\s{i} - \s{j} \geq 0$.
Therefore, it can be concluded that in either case, 
\begin{align}
    \gjk \cdot (\s{i} - \s{j}) \leq 0
\end{align}

Hence, we have 
\begin{align}
    \s{i} \cdot \gik - \fpik \leq  \gjk \leq 0
\end{align}

Since $\al{i}$ is linear bid and $i \in [n]$
\begin{align}
    \gapik = 0
\end{align}
\newpage

\section{Exercise 2}
\newcommand{\Lpn}{\ensuremath{L(p,n)}}
\newcommand{\Rpz}{\ensuremath{R(p,0)}}
\ovalbox{
\begin{minipage}{17cm}
    Let $p$ be a price vector for $G$ such that $L(p,n) \wedge R(p,0)$ holds.
    Prove that $\gapij = 0$ for all integers $i$ and $j$ in $[n]$. Hint: Make
    use of Exercise 1.
\end{minipage} } \\[0.5cm]

\begin{proof}[Case 1]
In the case of $i \in \{-1, n\}$ or in the case of $i \in [n]$ and $\al{i}$ is
single-item bid, it can be easy to conclude that $\forall i,j,\ \gapij = 0$ by definition of
$gap$ function.

\end{proof}

\begin{proof}[Case 2]
In the case of $i \in [n]$ and $\al{i}$ is linear bid, things are more
complicated. There are two subcases, (1) $i > j$, (2) $j > i$. We provide
detailed proof as follows.

Based on give condition $\Lpn \wedge \Rpz$, we have both of the followings
hold.
\begin{align}
    \Lpn \label{lpn} \\   
    \Rpz \label{rpz}
\end{align}

\noindent
{\it For $i > j$}, we can iteratively employ \eqref{rpz} and the result of exercise
1 to reach the conclusion of $\forall i,j,\ \gapij = 0$.
Specifically, apply the \eqref{rpz} repeatedly and since $\al{i}$ is linear
bid, it will eventually reach $\al{i}$. 

\begin{align}
    gap(p,\tau(j), j) &= 0 \label{start} \\
    gap(p,\tau(\tau(j)), \tau(j)) &= 0 \\
    \vdots& \nonumber \\
    gap(p,\tau_s(j),\tau_{s-1}(j)) &= 0 \\
    gap(p,i,\tau_s(j)) &= 0 \label{end}
\end{align}

Note that $s$ denotes the number of times we apply rule of $\Rpz$ and
$\tau_{s}(j)$ denotes the integer derived by apply $s$ times of $\Rpz$ on
original integer $j$. 

Then we can make use of the result of exercise 1 to address the series of
equations \eqref{start} - \eqref{end} and conclude that 

\begin{align}
    \forall i,j\ gap(p, i, j) = 0
\end{align}

Note that the above proof can be formalized by doing mathematical induction on
the number of times we employ $\Rpz$, which is $s$. \\[0.3cm]

\noindent
{\it For $i < j$}, we can iteratively employ \eqref{lpn} and the result of exercise
1 to reach the conclusion of $\forall i,j,\ \gapij = 0$. The detailed proof
follows the same routine as the case of $i > j$ in the above. 

In terms of the dicussion over two subcases, we have $\forall i,j,\ \gapij =
0$ in the case of $i \in [n]$ and $\al{i}$ is linear bid.

\end{proof}

According to the dicussion over {\it case 1} and {\it case 2}, it can be
concluded that {\it generally}
\begin{align}
    \forall i,j,\ \gapij = 0
\end{align}

%%%%%%%%%%%%%%%%%%%%%%%%%%%%%%%%%%%%%%%%%%%%%%%%%%%%%%%%%%%%%%%%%%%%%%%%
%%% General Documentation ends
%%%%%%%%%%%%%%%%%%%%%%%%%%%%%%%%%%%%%%%%%%%%%%%%%%%%%%%%%%%%%%%%%%%%%%%%
\end{document}
