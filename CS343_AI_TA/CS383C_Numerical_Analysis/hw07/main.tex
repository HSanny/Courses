%%%%%%%%%%%%%%%%%%%%%%%%%%%%%%%%%%%%%%%%%%%%%%%%%%%%%%%%%%%%%%%%%%%%%%%%
%%%  THIS TEX FILE IS TO GENERATE PDF FILE FOR 
%%% 
%%%  COPYRIGHT (C) JIMMY LIN, 2013, UT AUSTIN
%%%%%%%%%%%%%%%%%%%%%%%%%%%%%%%%%%%%%%%%%%%%%%%%%%%%%%%%%%%%%%%%%%%%%%%%
\documentclass[11pt,a4paper]{article}
%%%%%%%%%%%%%%%%%%%%%%%%%%%%%%%%%%%%%%%%%%%%%%%%%%%%%%%%%%%%%%%%%%%%%%%%
%%%  PACKAGES USED IN THIS TEX SOURCE FILE
%%%%%%%%%%%%%%%%%%%%%%%%%%%%%%%%%%%%%%%%%%%%%%%%%%%%%%%%%%%%%%%%%%%%%%%%
\usepackage{geometry,amsthm,amsmath,graphicx,fancyheadings,amsfonts,tikz}
\usepackage{tocloft}
\usepackage{../mcode}
\usepackage[colorlinks,
            linkcolor=blue,
            anchorcolor=red,
            citecolor=green
            ]{hyperref}
% for my mac
\IfFileExists{/Users/JimmyLin/.latex/UTA_CS/JS.sty}{ 
    \usepackage{/Users/JimmyLin/.latex/UTA_CS/JS}
    \usepackage{/Users/JimmyLin/.latex/UTA_CS/JSASGN}
}{} 
% for UT's linux machine
\IfFileExists{/u/jimmylin/workspace/Configs/latex/UTA_CS/JS.sty}{
    \usepackage{/u/jimmylin/.latex/UTA_CS/JS} 
    \usepackage{/u/jimmylin/.latex/UTA_CS/JSASGN}
}{} 
%%%%%%%%%%%%%%%%%%%%%%%%%%%%%%%%%%%%%%%%%%%%%%%%%%%%%%%%%%%%%%%%%%%%%%%%
%%% MACROS CONTAINING THE FILE INFORMATION
%%%%%%%%%%%%%%%%%%%%%%%%%%%%%%%%%%%%%%%%%%%%%%%%%%%%%%%%%%%%%%%%%%%%%%%%
\renewcommand{\COURSE}{CS383C Numerical Analysis}
\renewcommand{\LECTURER}{Robert A. van de Geijn}
\renewcommand{\SECTION}{53180}
\renewcommand{\TASK}{HW07 Cholesky Factorization}
\renewcommand{\RELEASEDATE}{Oct. 24  2014}
\renewcommand{\DUEDATE}{Oct. 30 2014}
\renewcommand{\TIMECONSUME}{2 hours}

\setlength{\cftsecnumwidth}{6.3em}
\renewcommand{\thesection}{Exercise \arabic{section}. }
\renewcommand{\thesubsection}{\arabic{section}.\arabic{subsection}}
\renewcommand{\contentsname}{Exercises}
%%%%%%%%%%%%%%%%%%%%%%%%%%%%%%%%%%%%%%%%%%%%%%%%%%%%%%%%%%%%%%%%%%%%%%%%
%%% DOCUMENTATION STARTS FROM HERE 
%%%%%%%%%%%%%%%%%%%%%%%%%%%%%%%%%%%%%%%%%%%%%%%%%%%%%%%%%%%%%%%%%%%%%%%%
\begin{document}
%%%%%%%%%%%%%%%%%%%%%%%%%%%%%%%%%%%%%%%%%%%%%%%%%%%%%%%%%%%%%%%%%%%%%%%%
%% TITLE PAGE
%%%%%%%%%%%%%%%%%%%%%%%%%%%%%%%%%%%%%%%%%%%%%%%%%%%%%%%%%%%%%%%%%%%%%%%%
\begin{titlepage}
    \maketitle
\end{titlepage}
%%%%%%%%%%%%%%%%%%%%%%%%%%%%%%%%%%%%%%%%%%%%%%%%%%%%%%%%%%%%%%%%%%%%%%%%
%% CONTENT PAGE: TABLEOFCONTENTS, LISTOFTABLES, LIST OF FIGURES
%%%%%%%%%%%%%%%%%%%%%%%%%%%%%%%%%%%%%%%%%%%%%%%%%%%%%%%%%%%%%%%%%%%%%%%%
\begin{center} 
    \tableofcontents  
%
%    %\listoftables 
%    %\listoffigures
\end{center}
%%%%%%%%%%%%%%%%%%%%%%%%%%%%%%%%%%%%%%%%%%%%%%%%%%%%%%%%%%%%%%%%%%%%%%%%
%%% GENERAL DOCUMENTATION BEGINS 
%%%%%%%%%%%%%%%%%%%%%%%%%%%%%%%%%%%%%%%%%%%%%%%%%%%%%%%%%%%%%%%%%%%%%%%%
\newpage
\setcounter{section}{2}
\section{Prove that $A = B^H B$ is HPD.}
To prove that $A = B^H B$ is HPD, we need to show both of the following:
\begin{itemize}
    \item $A^H = A$.
\begin{proof}
    \begin{align}
        A^H = (B^H B)^H = B^H (B^H)^H = B^H B = A
    \end{align}
\end{proof}
    \item $\forall x \not = 0,\ x^H A x > 0$.
\begin{proof}
    Let $x$ be arbitrary non-zero vector in $\mathbb{C}^n$
    \begin{align}
        x^H A x = x^H B^H B x = (Bx)^H Bx = || Bx ||_2^2 > 0
    \end{align}
    Note that for $x \not = 0$, $Bx = \sum_i B_i x_i \not = 0$, otherwise 
    $B$ is not linearly independent columns. 
\end{proof}
\end{itemize}
Since two properties above are proven, then we can conclude that 
\begin{center}
    If $B\in\mathbb{C}^{m\times n}$ has linearly independent columns, then $A
    = B^H B$ is HPD.
\end{center}

%%%%%%%%%%%%%%%%%%%%%%%%%%%%%%%%%%%%%%%%%%%%%%%%%%%%%%%%%%%%%%%%%%%%%%%%
\setcounter{section}{3}
\section{Show that diagonal elements are real and positive.}
\begin{itemize}
    \item diagonal elements of $A\in\mathbb{C}^{m\times m}$ are real.
        \begin{proof}
        Since $A$ is HPD, then $A^H = A$. Hence, for diagonal elements
        $\theta_0, \theta_1, ..., \theta_{m-1}$, then 
        \begin{align}
            \forall i = 0, ..., m-1,\ \theta_i^H = \theta_i
        \end{align}
        Let $\theta_i = x_i + y_i j$, where $j$ denotes imaginary unit, then
        \begin{align}
            \forall i = 0, ..., m-1,\ -y_i = y_i
        \end{align}
        That tells us 
        \begin{align}
            \forall i = 0, ..., m-1,\ y_i = 0
        \end{align}
        Then it can be concluded that 
       \begin{center} 
        all diagonal elements $\theta_0, \theta_1, ..., \theta_{m-1}$
        are real. (all imaginary part is zero.)
        \end{center}
        \end{proof}
    \item diagonal elements of $A\in\mathbb{C}^{m\times m}$ are positive.
        \begin{proof}
            Since $A\in\mathbb{C}^{m\times m}$ is HPD, then
            \begin{align}
                \forall x \not = 0,\ x^H A x > 0
            \end{align}
            Let $e_0$, ... $e_{m-1}$ denotes unit vector (whose 
            imaginary part is zero) that spans through the whole $\mathbb{C}^{m}$. 
            And let $\theta_0, \theta_1, ..., \theta_{m-1}$ denotes diagonal
            elements of HPD matrix $A$.
        \begin{align}
            \forall i = 0, ..., m-1,\ \theta_i = e_i^H A e_i > 0
        \end{align}
        Note that the $e_i^H A e_i \not = 0$ since $e_i \not = 0$. Hence, it
        can be conclude that 
       \begin{center} 
        all diagonal elements $\theta_0, \theta_1, ..., \theta_{m-1}$
        are positive. 
        \end{center}
        \end{proof}
\end{itemize}
%%%%%%%%%%%%%%%%%%%%%%%%%%%%%%%%%%%%%%%%%%%%%%%%%%%%%%%%%%%%%%%%%%%%%%%%
\newpage
\setcounter{section}{13}
\section{Implement Cholesky Factorization}
%\subsection{Unblocked version of Cholesky Factorization}
\lstinputlisting{../FLAME@lab/CHOL_unb.m}

%\newpage
%\subsection{Blocked version of Cholesky Factorization}
%\lstinputlisting{../FLAME@lab/CHOL_blk.m}

%%%%%%%%%%%%%%%%%%%%%%%%%%%%%%%%%%%%%%%%%%%%%%%%%%%%%%%%%%%%%%%%%%%%%%%%
\setcounter{section}{14}
\section{Relationship of Cholesky Factorization and QR}
\begin{proof}
    For matrix $B\in\mathbb{C}^{m\times n}$ with  linearly independent columns, it has an unique 
    QR factorization such that $B = QR$, where $Q\in \mathbb{C}^{m\times n}$
    and $R\in \mathbb{C}^{n\times n}$. And then for HPD matrix $A$, we have
    \begin{align}
       A = B^H B = (QR)^H QR = R^H Q^H Q R = \underbrace{R^H}_{L}
        \underbrace{R}_{L^H}
    \end{align}
\end{proof}
%%%%%%%%%%%%%%%%%%%%%%%%%%%%%%%%%%%%%%%%%%%%%%%%%%%%%%%%%%%%%%%%%%%%%%%%
%%% General Documentation ends
%%%%%%%%%%%%%%%%%%%%%%%%%%%%%%%%%%%%%%%%%%%%%%%%%%%%%%%%%%%%%%%%%%%%%%%%
\end{document}
