% -----------------------------*- LaTeX -*------------------------------
\documentclass[12pt]{report}
\usepackage{scribe_ee381v}
\begin{document}

% \course{CS 497}			% optional
% \coursetitle{Geometric Data Structures} % optional
% \semester{Fall 1998}			% optional
% \lecturer{Jeff Erickson}		% optional
\scribe{Jane A. Student}		% required
\lecturenumber{1}			% required, must be a number
\lecturedate{September 3}		% required, omit year

\maketitle

% ----------------------------------------------------------------------


\begin{danger}
This is the danger environment.
\end{danger}

\section{My first section heading}


\begin{theorem}
\label{ThmNeat}
This is a neat theorem.
\end{theorem}

\begin{proof}
Lengthy and technical proof.
\begin{equation}
\sum_{i=1}^n x_i = y.
\end{equation}
\end{proof}


\subsection{A subsection heading}


Here is how to typeset an array of equations.

\begin{eqnarray}
	x & = & y + z \\
%
     \alpha & = & \frac{\beta}{\gamma}
\end{eqnarray}



And a table.

\begin{table}[h]
\centerline{
    \begin{tabular}{|c|cc|}
	\hline
	\textbf{Method} & Cost & Iterations \\
	\hline
	Naive descent       & 12 & 200 \\
	Newton's method & 500 & 30 \\
	\hline
    \end{tabular}}
\caption{Comparison of different methods.}
\end{table}



\subsection{Yet another subsection}


\begin{corollary}
\label{CorThmNeat}
A corollary of Theorem~\ref{ThmNeat}.
\qed
\end{corollary}


\end{document}

