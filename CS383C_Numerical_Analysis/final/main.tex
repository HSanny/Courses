%%%%%%%%%%%%%%%%%%%%%%%%%%%%%%%%%%%%%%%%%%%%%%%%%%%%%%%%%%%%%%%%%%%%%%%%
%%%  THIS TEX FILE IS TO GENERATE PDF FILE FOR 
%%% 
%%%  COPYRIGHT (C) JIMMY LIN, 2013, UT AUSTIN
%%%%%%%%%%%%%%%%%%%%%%%%%%%%%%%%%%%%%%%%%%%%%%%%%%%%%%%%%%%%%%%%%%%%%%%%
\documentclass[11pt,a4paper]{article}
%%%%%%%%%%%%%%%%%%%%%%%%%%%%%%%%%%%%%%%%%%%%%%%%%%%%%%%%%%%%%%%%%%%%%%%%
%%%  PACKAGES USED IN THIS TEX SOURCE FILE
%%%%%%%%%%%%%%%%%%%%%%%%%%%%%%%%%%%%%%%%%%%%%%%%%%%%%%%%%%%%%%%%%%%%%%%%
\usepackage{geometry,amssymb,amsthm,amsmath,graphicx,fancyheadings,amsfonts,tikz}
\usepackage{../mcode}
\usepackage[colorlinks,
            linkcolor=blue,
            anchorcolor=red,
            citecolor=green
            ]{hyperref}
% for my mac
\IfFileExists{/Users/JimmyLin/.latex/UTA_CS/JS.sty}{ 
    \usepackage{/Users/JimmyLin/.latex/UTA_CS/JS}
    \usepackage{/Users/JimmyLin/.latex/UTA_CS/JSASGN}
}{} 
% for UT's linux machine
\IfFileExists{/u/jimmylin/workspace/Configs/latex/UTA_CS/JS.sty}{
    \usepackage{/u/jimmylin/.latex/UTA_CS/JS} 
    \usepackage{/u/jimmylin/.latex/UTA_CS/JSASGN}
}{} 
%%%%%%%%%%%%%%%%%%%%%%%%%%%%%%%%%%%%%%%%%%%%%%%%%%%%%%%%%%%%%%%%%%%%%%%%
%%% MACROS CONTAINING THE FILE INFORMATION
%%%%%%%%%%%%%%%%%%%%%%%%%%%%%%%%%%%%%%%%%%%%%%%%%%%%%%%%%%%%%%%%%%%%%%%%
\renewcommand{\COURSE}{CS383C Numerical Analysis}
\renewcommand{\LECTURER}{Robert A. van de Geijn}
\renewcommand{\SECTION}{53180}
\renewcommand{\TASK}{Final Exam}
\renewcommand{\RELEASEDATE}{Dec. 2 2014}
\renewcommand{\DUEDATE}{Dec. 10 2014}
\renewcommand{\TIMECONSUME}{10 hours}

%\renewcommand{\thesection}{Exercise \arabic{section}.}
%\renewcommand{\thesubsection}{(\alph{subsection})}
\renewcommand{\contentsname}{Exercises}

%%%%%%%%%%%%%%%%%%%%%%%%%%%%%%%%%%%%%%%%%%%%%%%%%%%%%%%%%%%%%%%%%%%%%%%%
%%% DOCUMENTATION STARTS FROM HERE 
%%%%%%%%%%%%%%%%%%%%%%%%%%%%%%%%%%%%%%%%%%%%%%%%%%%%%%%%%%%%%%%%%%%%%%%%
\begin{document}
%%%%%%%%%%%%%%%%%%%%%%%%%%%%%%%%%%%%%%%%%%%%%%%%%%%%%%%%%%%%%%%%%%%%%%%%
%% TITLE PAGE
%%%%%%%%%%%%%%%%%%%%%%%%%%%%%%%%%%%%%%%%%%%%%%%%%%%%%%%%%%%%%%%%%%%%%%%%
\begin{titlepage}
    \maketitle
\end{titlepage}
%%%%%%%%%%%%%%%%%%%%%%%%%%%%%%%%%%%%%%%%%%%%%%%%%%%%%%%%%%%%%%%%%%%%%%%%
%% CONTENT PAGE: TABLEOFCONTENTS, LISTOFTABLES, LIST OF FIGURES
%%%%%%%%%%%%%%%%%%%%%%%%%%%%%%%%%%%%%%%%%%%%%%%%%%%%%%%%%%%%%%%%%%%%%%%%
\begin{center} 
    \tableofcontents  
%    %\listoftables 
%    %\listoffigures
\end{center}
%%%%%%%%%%%%%%%%%%%%%%%%%%%%%%%%%%%%%%%%%%%%%%%%%%%%%%%%%%%%%%%%%%%%%%%%
%%% GENERAL DOCUMENTATION BEGINS 
%%%%%%%%%%%%%%%%%%%%%%%%%%%%%%%%%%%%%%%%%%%%%%%%%%%%%%%%%%%%%%%%%%%%%%%%
\newpage
\part{Cholesky Factorization}
\section{SPD}
\renewcommand{\AA}{\left( \begin{array}{cc}
            A_{00} & a_{10}  \\
            a_{10}^T & \alpha_{11} 
        \end{array} \right) }
\newcommand{\AAT}{\left( \begin{array}{cc}
            A_{00}^T & a_{10}  \\
            a_{10}^T & \alpha_{11}
        \end{array} \right) }
\newcommand{\xx}{\left( \begin{array}{c}
            x_{0}  \\ \chi_{1} 
        \end{array} \right)}
\subsection{Show $A_{00}$ is SPD}
\begin{proof}
    Since $A$ is SPD, then
    \begin{align}
        A^T = A \label{sym} \\
        \forall x,\ x^T A x \geq 0 \label{pd}
    \end{align}
    From \eqref{sym}, we have
    \begin{align}
        A^T = \AA^T = \AAT = A
    \end{align}
    Then 
    \begin{align} \label{symx0}
        A_{00}^T = A_{00} \tag{symmetry of A}
    \end{align}
    Also, we denote arbitrary $x = \left( \begin{array}{c}
            x_{0}  \\ \chi_{1} 
        \end{array} \right)$ and then from \eqref{pd}, we have
    \begin{align}
        x^T A x = \xx^T \AA \xx =  x_0^T A_{00} x_0 + 2\chi_1 a_{10}^T
        x_0 + \chi_1^2 \alpha_{11}  > 0,\ \forall x_0, \chi_1
    \end{align}
    Let $\chi_1 = 0$, then we have
    \begin{align} \label{pdx0}
        x_0^T A_{00} x_0  > 0,\ \forall x_0 \tag{positive definiteness}
    \end{align}
    In terms of \eqref{symx0} and \eqref{pdx0}, then it is proved that
    $A_{00}$ is SPD.
\end{proof}
\subsection{$l_{10}^T = a_{10}^T L_{00}^{-T}$ is well defined}
Since $L_{00}$ is non-singular, then it is easy to derive that $L_{00}^T$ is also non-singluar.
Then $L_{00}^{-T}$ exists. Hence,
\begin{align}
    l_{10}^T = a_{10}^T L_{00}^{-T}
\end{align}
is well-defined. 

\subsection{$\alpha_{11} - l_{10}^T l_{10} > 0$}



\subsection{Show equality}
\newcommand{\LL}{\left( \begin{array}{cc}
            L_{00} & l_{10}  \\
            l_{10}^T & \lambda_{11} 
        \end{array} \right) }
\begin{align}
   L\cdot L^T =  \LL \LL^T = \left( \begin{array}{cc}
            L_{00} L_{00}^T & L_{00} l_{10}  \\
            l_{10}^T L_{00}^T & l_{10}^T l_{10} + \lambda_{11}^2
        \end{array} \right) 
\end{align} 
Obviously, $L \cdot L^T = A$ if only if 
\begin{align}
    A_{00} &= L_{00} L_{00}^T \\
    a_{10} &= L_{00} l_{10} \\
    \alpha_{11} &= L_{10}^T l_{10} + \lambda_{11} ^2
\end{align}
\section{}
\subsection{Another proof of Cholesky Factorization Theorem}

\subsection{Bordered Cholesky Algorithm}


\section{Cost of Bordered Algorithm}

\newpage
\part{Method of Relatively Robust Representations}
\setcounter{section}{0}
\section{$LDL^T$ Factorization for indefinite matrices}
\lstinputlisting{../FLAME@lab/LDL_unb.m}
\newpage

\section{$LDL^T$ Factorization for tridiagonal matrices}
\textbf{Codes}:
\lstinputlisting{../FLAME@lab/LDL_TRI_unb.m}

\noindent
\textbf{Costs}: 
\begin{itemize}
    \item Divide: $ 1 \cdot n = n$ (l21 update)  
    \item Multiply: $1 \cdot n = n$ (alpha22 update)
    \item Add/Subtract: $1 \cdot n = n$ (alpha21 update) 
\end{itemize}
In terms of above analysis, the approximate cost is $\BigO{n}$. \\

\noindent
\textbf{Analytics}: 
    The way I come up with this algorithm is to instantiate the $LDL^T$
    factorization in last question to the case of tridiagonal matrices. That
    is, treat the $alpha21$ and $l21$ as vectors with only one non-zero entry.

\newpage
\section{$UDU^T$ Factorization for indefinite matrices}
\lstinputlisting{../FLAME@lab/UDU_unb.m}

\newpage
\section{$UDU^T$ Factorization for tridiagonal matrices}
\lstinputlisting{../FLAME@lab/UDU_TRI_unb.m}

\newpage
\section{Twisted Factorization: $\phi_1$}
\newcommand{\AAA}{\left( \begin{array}{ccc}
        A_{00} & \alpha_{10} e_L & 0 \\
        \alpha_{10} e_L^T & \alpha_{11} & \alpha_{21} e_F^T \\
        0 & \alpha_{21} e_F & A_{22}
    \end{array} \right)}
\newcommand{\LLL}{\left( \begin{array}{ccc}
        L_{00} & 0 & 0 \\
        \lambda_{10} e_L^T & 1 & 0 \\
        0 & \lambda_{21} e_F & L_{22}
    \end{array} \right)}
 \newcommand{\UUU}{\left( \begin{array}{ccc}
             U_{00} & v_{01}e_L & 0 \\
             0 & 1 & v_{21} e_F^T \\
             0 & 0 & U_{22}
    \end{array} \right)}
\newcommand{\DDD}{\left( \begin{array}{ccc}
        D_{00} & 0 & 0 \\
        0 & \delta_1 & 0 \\
        0 & 0 & D_{22}
    \end{array} \right)}
\newcommand{\EEE}{\left( \begin{array}{ccc}
        E_{00} & 0 & 0 \\
        0 & \epsilon_1 & 0 \\
        0 & 0 & E_{22}
    \end{array} \right)}
For $LDL^T$ factorization, we have 
\begin{align}
    A &= L D L^T \nonumber \\
    &= \LLL \DDD \LLL^T  \nonumber \\
    &= \left( \begin{array}{ccc}
            L_{00} D_{00} L_{00}^T & \lambda_{10} L_{00} D_{00} e_L & 0 \\
        \lambda_{10} e_L^T D_{00} L_{00}^T & \lambda_{10}^2 e_L^T D_{00} e_L +
        \delta_1 & \lambda_{21} \delta_1 e_F^T \\
        0 & \lambda_{21} \delta_{1} e_F & \lambda_{21}^2 \delta_1 e_F e_F^T +
        L_{22} D_{22} L_{22}^T \\
    \end{array} \right) \\
    & = \AAA
\end{align}
And by matching, we have 
\begin{equation} \label{111}
    \begin{aligned}
    A_{00}  &= L_{00} D_{00} L_{00}^T  \\
   \alpha_{10} e_L  &= \lambda_{10} L_{00} D_{00} e_L \\
    \alpha_{11} &= \lambda_{10}^2 e_L^T D_{00} e_L + \delta_1 \\
    \alpha_{21} &= \lambda_{21} \delta_1\\
    A_{22} &= \lambda_{21}^2 \delta_1 e_F e_F^T + L_{22} D_{22} L_{22}^T
    \end{aligned}
\end{equation}
Similarly, for $UEU^T$ factorization, we have
\begin{align}
    A &= U E U^T \nonumber \\
      &= \UUU \EEE \UUU^T  \nonumber\\
      &= \left( \begin{array}{ccc}
              U_{00} E_{00} U_{00}^T + v_{01} \epsilon_{1} e_L e_L^T & v_{01} \epsilon_1 e_L & 0 \\
              v_{01} \epsilon_1 e_L^T & v_{21}^2 e_F^T E_{22} e_F + \epsilon_1 
        & v_{21} e_F^T E_{22} U_{22}^T \\
        0 & v_{21} U_{22} E_{22} e_F & U_{22} E_{22} U_{22}^T
    \end{array} \right) \\
    &= \AAA
\end{align}
And by matching, we have 
\begin{equation} \label{222}
    \begin{aligned}
    A_{00}  &= U_{00} E_{00} U_{00}^T + v_{01} \epsilon_{1} e_L e_L^T  \\
    \alpha_{10}  &= v_{01} \epsilon_1   \\
    \alpha_{11} &= v_{21}^2 e_F^T E_{22} e_F + \epsilon_1 \\
    \alpha_{21} e_F^T &= v_{21} e_F^T E_{22} U_{22}^T \\
    A_{22} &= U_{22} E_{22} U_{22}^T
\end{aligned}
\end{equation}
Now we consider the Twisted Factorization
\begin{align}
    & \hspace{0.5cm} \left( \begin{array}{ccc}
        L_{00} & 0 & 0 \\
        \lambda_{10} e_L^T & 1 & v_{21} e_F^T \\
        0 & 0 & U_{22}
    \end{array} \right)
   \left( \begin{array}{ccc}
        D_{00} & 0 & 0 \\
        0 & \phi_1 & 0 \\
        0 & 0 & D_{22}
    \end{array} \right) 
\left( \begin{array}{ccc}
        L_{00} & 0 & 0 \\
        \lambda_{10} e_L^T & 1 & v_{21} e_F^T \\
        0 & 0 & U_{22}
    \end{array} \right)^T \\
 &= 
 \left( \begin{array}{ccc}
         L_{00} D_{00} L_{00}^T & \lambda_{10} L_{00} D_{00} e_L & 0 \\
        \lambda_{10} e_L^T D_{00} L_{00}^T & \phi_1 + \lambda_{10}^2 e_L^T D_{00} e_L + v_{21}^2 e_F^T E_{22} e_F & v_{21} e_F^T E_{22} U_{22}^T \\
        0 & v_{21} U_{22} E_{22} e_F & U_{22} E_{22} U_{22}^T 
    \end{array} \right) \\
  &= \AAA
\end{align}
Then we have
\begin{align} \label{333}
    \alpha_{11} = \phi_1 + \lambda_{10}^2 e_L^T D_{00} e_L + v_{21}^2 e_F^T E_{22} e_F 
\end{align}
To satisfy \eqref{111}, \eqref{222} and \eqref{333} at the same time, we need to have 
\begin{align}
    \phi_1 = \frac{\delta_1 + \epsilon_1 - \lambda_{10}^2 e_L^T D_{00} e_L - v_{21}^2 e_F^T E_{22} e_F }{2}
\end{align}

\noindent
\textbf{Complexity}: 
\begin{itemize} 
    \item computation of $e_L^T D_{00} e_L$ or $e_F^T E_{22} e_F$ is
        $\BigO{1}$. (constant time)
    \item computation of the factorized matrix, it requires $\BigO{n}$ for
         assembling components of $U$ and $L$ so as to derive the resulted matrix.
\end{itemize}

%%%%%%%%%%%%%%%%%%%%%%%%%%%%%%%%%%%%%%%%%%%%%%%%%%%%%%%%%%%%%%%%%%%%%%%%
\newpage
\section{Twisted Factorization: Eigenvector}
\renewcommand{\LLL}{\left( \begin{array}{ccc}
        L_{00} & 0 & 0 \\
        \lambda_{10} e_L^T & 1 & v_{21} e_F^T \\
        0 & 0 & U_{22} \end{array} \right)}
\renewcommand{\DDD}{\left( \begin{array}{ccc}
        D_{00} & 0 & 0 \\
        0 & 0 & 0 \\
        0 & 0 & E_{22} \end{array} \right)}
\newcommand{\xxx}{\left( \begin{array}{c}
        x_{0}  \\
        \chi_1   \\
        x_2  \end{array} \right)}
\newcommand{\yyy}{\left( \begin{array}{c}
        y_{0}  \\
        \psi_1   \\
        y_2  \end{array} \right)}
\newcommand{\zerovec}{\left( \begin{array}{c}
        0 \\
        0 \\ 
        0  \end{array} \right)}
Separate terms of the known condition as follows:
\begin{align}
    \underbrace{\LLL \DDD}_{S} \underbrace{\LLL^T \xxx}_y = \zerovec
\end{align}
Then we derive the form of $S$
\begin{align}
   S \triangleq \LLL \cdot \DDD = \left( \begin{array}{ccc}
            L_{00} D_{00} & 0 & 0 \\
            \lambda_{10} e_L^T D_{00} & 0 & v_{21} e_F^T E_{22} \\
            0 & 0 & U_{22} E_{22} \end{array} \right)
\end{align}
Then relate it to $y$
\begin{align} \label{444}
    S \cdot y = \left( \begin{array}{ccc}
            L_{00} D_{00} & 0 & 0 \\
            \lambda_{10} e_L^T D_{00} & 0 & v_{21} e_F^T E_{22} \\
            0 & 0 & U_{22} E_{22} \end{array} \right) \yyy = \zerovec
\end{align}
where 
\begin{equation} \label{555}
    \begin{aligned}
    L_{00}^T x_{0} + \lambda_{10} e_L \chi_1 &= y_0 \\
    \chi_1 &= \psi_1  \\
    v_{21}\chi_1 e_F + U_{22}^T x_{2} &= y_2
    \end{aligned}
\end{equation}
Solve \eqref{444}, we have
\begin{equation} 
    \begin{aligned}
    y_0 &= 0 \\
    \psi_1 &= c \text{ (constant) } \\
    y_2 &= 0
    \end{aligned}
\end{equation}
In terms of \eqref{555}, for the vector $x$, we need to solve the following system
\begin{equation} 
    \begin{aligned}
    L_{00}^T x_{0} + \lambda_{10} e_L \chi_1 &= 0 \\
    \chi_1 &= c  \\
    v_{21}\chi_1 e_F + U_{22}^T x_{2} &= 0
    \end{aligned}
\end{equation}
Note that this equation system has infinity number of solutions unless we set
$c$ fixed. Here, we set $c = 1$ for simplicity. Then
\begin{align}
    L_{00}^T x_{0} &= - \lambda_{10} e_L   \\
    U_{22}^T x_{2} &= - v_{21} e_F 
\end{align}
which is actually two gaussian elimination problem. In terms of the special
structure of $L_{00}$ and $U_{22}$, the solution takes complexity $\BigO{n^2}$.

%%%%%%%%%%%%%%%%%%%%%%%%%%%%%%%%%%%%%%%%%%%%%%%%%%%%%%%%%%%%%%%%%%%%%%%%
%%% General Documentation ends
%%%%%%%%%%%%%%%%%%%%%%%%%%%%%%%%%%%%%%%%%%%%%%%%%%%%%%%%%%%%%%%%%%%%%%%%
\end{document}
