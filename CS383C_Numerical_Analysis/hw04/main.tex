%%%%%%%%%%%%%%%%%%%%%%%%%%%%%%%%%%%%%%%%%%%%%%%%%%%%%%%%%%%%%%%%%%%%%%%%
%%%  THIS TEX FILE IS TO GENERATE PDF FILE FOR 
%%% 
%%%  COPYRIGHT (C) JIMMY LIN, 2013, UT AUSTIN
%%%%%%%%%%%%%%%%%%%%%%%%%%%%%%%%%%%%%%%%%%%%%%%%%%%%%%%%%%%%%%%%%%%%%%%%
\documentclass[11pt,a4paper]{article}
%%%%%%%%%%%%%%%%%%%%%%%%%%%%%%%%%%%%%%%%%%%%%%%%%%%%%%%%%%%%%%%%%%%%%%%%
%%%  PACKAGES USED IN THIS TEX SOURCE FILE
%%%%%%%%%%%%%%%%%%%%%%%%%%%%%%%%%%%%%%%%%%%%%%%%%%%%%%%%%%%%%%%%%%%%%%%%
\usepackage{geometry,amsthm,amsmath,graphicx,fancyheadings,amsfonts,tikz}
\usepackage[colorlinks,
            linkcolor=blue,
            anchorcolor=red,
            citecolor=green
            ]{hyperref}
% for my mac
\IfFileExists{/Users/JimmyLin/.latex/UTA_CS/JS.sty}{ 
    \usepackage{/Users/JimmyLin/.latex/UTA_CS/JS}
    \usepackage{/Users/JimmyLin/.latex/UTA_CS/JSASGN}
}{} 
% for UT's linux machine
\IfFileExists{/u/jimmylin/workspace/Configs/latex/UTA_CS/JS.sty}{
    \usepackage{/u/jimmylin/.latex/UTA_CS/JS} 
    \usepackage{/u/jimmylin/.latex/UTA_CS/JSASGN}
}{} 
%%%%%%%%%%%%%%%%%%%%%%%%%%%%%%%%%%%%%%%%%%%%%%%%%%%%%%%%%%%%%%%%%%%%%%%%
%%% MACROS CONTAINING THE FILE INFORMATION
%%%%%%%%%%%%%%%%%%%%%%%%%%%%%%%%%%%%%%%%%%%%%%%%%%%%%%%%%%%%%%%%%%%%%%%%
\renewcommand{\COURSE}{CS383C Numerical Analysis}
\renewcommand{\LECTURER}{Robert A. Van De Geijn}
\renewcommand{\SECTION}{53180}
\renewcommand{\TASK}{Homework 04}
\renewcommand{\RELEASEDATE}{Oct. 02 2014}
\renewcommand{\DUEDATE}{Oct. 09 2014}
\renewcommand{\TIMECONSUME}{10 hours}

\renewcommand{\thesection}{Exercise \arabic{section}.}
\renewcommand{\thesubsection}{\arabic{section}.\arabic{subsection}}
\renewcommand{\contentsname}{Exercises}
%%%%%%%%%%%%%%%%%%%%%%%%%%%%%%%%%%%%%%%%%%%%%%%%%%%%%%%%%%%%%%%%%%%%%%%%
%%% DOCUMENTATION STARTS FROM HERE 
%%%%%%%%%%%%%%%%%%%%%%%%%%%%%%%%%%%%%%%%%%%%%%%%%%%%%%%%%%%%%%%%%%%%%%%%
\begin{document}
%%%%%%%%%%%%%%%%%%%%%%%%%%%%%%%%%%%%%%%%%%%%%%%%%%%%%%%%%%%%%%%%%%%%%%%%
%% TITLE PAGE
%%%%%%%%%%%%%%%%%%%%%%%%%%%%%%%%%%%%%%%%%%%%%%%%%%%%%%%%%%%%%%%%%%%%%%%%
\begin{titlepage}
    \maketitle
\end{titlepage}
%%%%%%%%%%%%%%%%%%%%%%%%%%%%%%%%%%%%%%%%%%%%%%%%%%%%%%%%%%%%%%%%%%%%%%%%
%% CONTENT PAGE: TABLEOFCONTENTS, LISTOFTABLES, LIST OF FIGURES
%%%%%%%%%%%%%%%%%%%%%%%%%%%%%%%%%%%%%%%%%%%%%%%%%%%%%%%%%%%%%%%%%%%%%%%%
\begin{center} 
    \tableofcontents  
%
%    %\listoftables 
%    %\listoffigures
\end{center}
%%%%%%%%%%%%%%%%%%%%%%%%%%%%%%%%%%%%%%%%%%%%%%%%%%%%%%%%%%%%%%%%%%%%%%%%
%%% GENERAL DOCUMENTATION BEGINS 
%%%%%%%%%%%%%%%%%%%%%%%%%%%%%%%%%%%%%%%%%%%%%%%%%%%%%%%%%%%%%%%%%%%%%%%%
\newpage
\part{Exercises on Solving LLS Problems}
\setcounter{section}{1}
\section{}

\section{}

\section{}
%%%%%%%%%%%%%%%%%%%%%%%%%%%%%%%%%%%%%%%%%%%%%%%%%%%%%%%%%%%%%%%%%%%%%%%%

%%%%%%%%%%%%%%%%%%%%%%%%%%%%%%%%%%%%%%%%%%%%%%%%%%%%%%%%%%%%%%%%%%%%%%%%
\newpage
\part{Exercises on Conditioning}
\setcounter{section}{0}
\section{}
Show that, for {\it a consistent matrix norm}, $\kappa(A) \geq 1$.

\begin{proof}
\begin{align}
    \kappa(A) = ||A|| \cdot ||A^{-1}|| \geq || A A^{-1}|| = ||I|| = 1
\end{align}
Note that the above $||\cdot||$ was for arbitrary induced matrix norm.
\end{proof}

\begin{lemma}
    For arbitrary matrix $A$ and $B$, $|| AB || \leq || A || \cdot || B ||$.
\end{lemma}
\begin{proof}
    \begin{align}
        || A B || = \sup_{x \not =0} \frac{|| A B x ||}{|| x ||} 
        &= \sup_{x \not =0} \frac{|| A (B x) ||}{|| x ||} \\ 
        &\leq \sup_{x \not =0} \frac{|| A || \cdot || B x ||}{|| x ||} \\
        &\leq \sup_{x \not =0} \frac{|| A || \cdot || B || \cdot || x || }{|| x ||} \\
        &= || A || \cdot || B || 
\end{align}
Hence, it is concluded that $|| A B || \leq || A || \cdot || B ||$.
\end{proof}

\begin{lemma}
    For abitrary norm $|| \cdot ||$ and identity matrix I,\ $|| I || = 1$.
\end{lemma}
\begin{proof}
    \begin{align}
        || I || = \sup_{x\not =0} \frac{|| I \cdot x ||}{|| x ||} 
        = \sup_{x\not =0} \frac{|| x ||}{|| x ||} 
        = 1
    \end{align}
\end{proof}

%%%%%%%%%%%%%%%%%%%%%%%%%%%%%%%%%%%%%%%%%%%%%%%%%%%%%%%%%%%%%%%%%%%%%%%%
\newpage
\section{}
If $A$ has lineraly independent columns, show that 
$|| (A^H A)^{-1} A^H ||_2 = \frac{1}{\sigma_{n-1}}$, where $\sigma_{n-1}$
equals {\it the smallest sigular value} of $A$. 

\begin{proof}
    Let $U$, $\Sigma$ and $V$ be singular value decomposition of $A$, such
    that $A = U\Sigma V^H$.

    \begin{align}
         || (A^H A)^{-1} A^H ||_2 
        &= || \big( (U\Sigma V^H)^H U\Sigma V^H \big)^{-1} (U\Sigma V^H)^H ||_2 \\
        &= || \big( V \Sigma^H U^H U\Sigma V^H \big)^{-1} V \Sigma^H U^H ||_2 \\
        &= || \big( V \Sigma^H \Sigma V^H \big)^{-1} V \Sigma^H U^H ||_2 \\
        &= || V^{-H} \Sigma^{-1} \Sigma^{-H} V^{-1}  V \Sigma^H U^H ||_2 \\
        &= || V^{-H} \Sigma^{-1} \Sigma^{-H} \Sigma^H U^H ||_2 \\
        &= || V^{-H} \Sigma^{-1} U^H ||_2 \\
        &= || V \Sigma^{-1} U^H ||_2 \\
        &= || \Sigma^{-1}||_2 \\
        &= \frac{1}{\sigma_{n-1}}
    \end{align}
\end{proof}

\begin{lemma}
    (Unitary Invariance) For arbitrary unitary matrix $U$, 
    \begin{align}
        || U A ||_2 = || A U ||_2 = || A ||_2 
    \end{align}
\end{lemma}

\begin{lemma}
    For arbitrary diagonal matrix $\Sigma$, 
    \begin{align}
        || \Sigma^{-1} ||_2 = \frac{1}{\sigma_{n-1}}
    \end{align}
    where, $\sigma_{n-1}$ is the least entry of $\Sigma$. 
\end{lemma}
Note that above two lemmas have been proven in exercises of previou notes.

%%%%%%%%%%%%%%%%%%%%%%%%%%%%%%%%%%%%%%%%%%%%%%%%%%%%%%%%%%%%%%%%%%%%%%%%
\newpage
\section{}
Let $A$ have linearly independent columns. Show that $\kappa_2 (A^H A) =
\kappa_2 (A)^2$ .
\begin{proof}
    We achieve the proof by employing SVD over $A$. Let unitary matrix $U$,
    diagonal matrix $\Sigma$ and unitary matrix $V$ be singular value
    decomposition of $A$, such that $A = U \Sigma V^H$. We start from the
    definition of condition number $\kappa_2 (\cdot)$. 
    \begin{align}
        \kappa_2 (A^H A) &= || A^H A ||_2 \cdot || (A^H A)^{-1} ||_2 
    \end{align}
    Then we discuss the term $|| A^H A ||_2$ and $|| (A^H A)^{-1} ||_2$ respectively.
    \begin{align}
        || A^H A ||_2 
        &=  || (U \Sigma V^H)^H U \Sigma V^H ||_2 \\
        &=  || V \Sigma^H U^H U \Sigma V^H ||_2 \\
        &=  || V \Sigma^H \Sigma V^H ||_2 \\
        &=  || \Sigma^H \Sigma ||_2 \\
        &=  \sigma^2_0 \\
        &= || A ||_2^2
    \end{align}
    Note that $\sigma_0$ is the largest singular value of matrix $A$ and also
    the largest entry of $\Sigma$.
 \begin{align}
     || (A^H A)^{-1} ||_2 
     &=  || \big( (U \Sigma V^H)^H U \Sigma V^H \big)^{-1} ||_2 \\
     &=  || \big( V \Sigma^H U^H U \Sigma V^H \big)^{-1} ||_2 \\
     &=  || \big( V \Sigma^H \Sigma V^H \big)^{-1} ||_2 \\
     &=  || V^{-H} \Sigma^{-1} \Sigma^{-H} V^{-1}  ||_2 \\
     &=  || \Sigma^{-1} \Sigma^{-H} ||_2 \\
     &=  || \Sigma^{-1} \Sigma^{-1} ||_2 \\
     &=  \frac{1}{\sigma_{n-1}^2} \\
     &= || A^{-1} ||_2^2
    \end{align}   
    Now we have 
    \begin{align}
        || A^H A ||_2 &= || A ||_2^2 \\
     || (A^H A)^{-1} ||_2 &= || A^{-1} ||_2^2
    \end{align}
    Then
    \begin{align}
        \kappa_2 (A^H A) &= || A^H A ||_2 \cdot || (A^H A)^{-1} ||_2  \\
        & = || A ||_2^2 \cdot || A^{-1} ||_2^2 \\
        & = \big( || A ||_2 \cdot || A^{-1} ||_2 \big)^2 \\
        & = \kappa_2(A)^2
    \end{align}
    Hence, it can be concluded that 
    \begin{align}
        \kappa_2 (A^H A) & = \kappa_2(A)^2
    \end{align}
\end{proof}

%%%%%%%%%%%%%%%%%%%%%%%%%%%%%%%%%%%%%%%%%%%%%%%%%%%%%%%%%%%%%%%%%%%%%%%%
\newpage
\section{}

%%%%%%%%%%%%%%%%%%%%%%%%%%%%%%%%%%%%%%%%%%%%%%%%%%%%%%%%%%%%%%%%%%%%%%%%
\newpage
\section{}

Let $U \in \mathbb{C}^{n\times n}$ be unitary. Show that $\kappa_2(U) = 1$.
\begin{proof}
    \begin{align}
        \kappa_2 (U) &= || U ||_2 || U^{-1} ||_2  \\
        & = \sup_{x\not = 0} \frac{|| U x ||_2}{|| x ||_2} \cdot
        \sup_{y\not = 0} \frac{|| U^{-1} y ||_2}{|| y ||_2} \\
        & = \sup_{x\not = 0} \frac{|| x ||_2}{|| x ||_2} \cdot
        \sup_{y\not = 0} \frac{|| y ||_2}{|| y ||_2} \\
        & = 1 \cdot 1 \\
        & = 1
    \end{align}
\end{proof}

\begin{lemma}
    For arbitrary unitary matrix $U$, its inverse $U^{-1}$ is still unitary. 
\end{lemma}
%%%%%%%%%%%%%%%%%%%%%%%%%%%%%%%%%%%%%%%%%%%%%%%%%%%%%%%%%%%%%%%%%%%%%%%%
%%% General Documentation ends
%%%%%%%%%%%%%%%%%%%%%%%%%%%%%%%%%%%%%%%%%%%%%%%%%%%%%%%%%%%%%%%%%%%%%%%%
\end{document}
