%%%%%%%%%%%%%%%%%%%%%%%%%%%%%%%%%%%%%%%%%%%%%%%%%%%%%%%%%%%%%%%%%%%%%%%%
%%%  THIS TEX FILE IS TO GENERATE PDF FILE FOR 
%%% 
%%%  COPYRIGHT (C) JIMMY LIN, 2013, UT AUSTIN
%%%%%%%%%%%%%%%%%%%%%%%%%%%%%%%%%%%%%%%%%%%%%%%%%%%%%%%%%%%%%%%%%%%%%%%%
\documentclass[11pt,a4paper]{article}
%%%%%%%%%%%%%%%%%%%%%%%%%%%%%%%%%%%%%%%%%%%%%%%%%%%%%%%%%%%%%%%%%%%%%%%%
%%%  PACKAGES USED IN THIS TEX SOURCE FILE
%%%%%%%%%%%%%%%%%%%%%%%%%%%%%%%%%%%%%%%%%%%%%%%%%%%%%%%%%%%%%%%%%%%%%%%%
\usepackage{geometry,amsthm,amsmath,graphicx,fancyheadings,amsfonts,tikz}
\usepackage{../mcode}
\usepackage[colorlinks,
            linkcolor=blue,
            anchorcolor=red,
            citecolor=green
            ]{hyperref}
% for my mac
\IfFileExists{/Users/JimmyLin/.latex/UTA_CS/JS.sty}{ 
    \usepackage{/Users/JimmyLin/.latex/UTA_CS/JS}
    \usepackage{/Users/JimmyLin/.latex/UTA_CS/JSASGN}
}{} 
% for UT's linux machine
\IfFileExists{/u/jimmylin/workspace/Configs/latex/UTA_CS/JS.sty}{
    \usepackage{/u/jimmylin/.latex/UTA_CS/JS} 
    \usepackage{/u/jimmylin/.latex/UTA_CS/JSASGN}
}{} 
%%%%%%%%%%%%%%%%%%%%%%%%%%%%%%%%%%%%%%%%%%%%%%%%%%%%%%%%%%%%%%%%%%%%%%%%
%%% MACROS CONTAINING THE FILE INFORMATION
%%%%%%%%%%%%%%%%%%%%%%%%%%%%%%%%%%%%%%%%%%%%%%%%%%%%%%%%%%%%%%%%%%%%%%%%
\renewcommand{\COURSE}{CS383C Numerical Analysis}
\renewcommand{\LECTURER}{Robert A. van de Geijn}
\renewcommand{\SECTION}{53180}
\renewcommand{\TASK}{HW06 LU Factorization}
\renewcommand{\RELEASEDATE}{Oct.  2014}
\renewcommand{\DUEDATE}{Oct.  2014}
\renewcommand{\TIMECONSUME}{10 hours}

\renewcommand{\thesection}{Exercise \arabic{section}.}
\renewcommand{\thesubsection}{\arabic{section}.\arabic{subsection}}
\renewcommand{\contentsname}{Exercises}
%%%%%%%%%%%%%%%%%%%%%%%%%%%%%%%%%%%%%%%%%%%%%%%%%%%%%%%%%%%%%%%%%%%%%%%%
%%% DOCUMENTATION STARTS FROM HERE 
%%%%%%%%%%%%%%%%%%%%%%%%%%%%%%%%%%%%%%%%%%%%%%%%%%%%%%%%%%%%%%%%%%%%%%%%
\begin{document}
%%%%%%%%%%%%%%%%%%%%%%%%%%%%%%%%%%%%%%%%%%%%%%%%%%%%%%%%%%%%%%%%%%%%%%%%
%% TITLE PAGE
%%%%%%%%%%%%%%%%%%%%%%%%%%%%%%%%%%%%%%%%%%%%%%%%%%%%%%%%%%%%%%%%%%%%%%%%
\begin{titlepage}
    \maketitle
\end{titlepage}
%%%%%%%%%%%%%%%%%%%%%%%%%%%%%%%%%%%%%%%%%%%%%%%%%%%%%%%%%%%%%%%%%%%%%%%%
%% CONTENT PAGE: TABLEOFCONTENTS, LISTOFTABLES, LIST OF FIGURES
%%%%%%%%%%%%%%%%%%%%%%%%%%%%%%%%%%%%%%%%%%%%%%%%%%%%%%%%%%%%%%%%%%%%%%%%
\begin{center} 
    \tableofcontents  
%
%    %\listoftables 
%    %\listoffigures
\end{center}
%%%%%%%%%%%%%%%%%%%%%%%%%%%%%%%%%%%%%%%%%%%%%%%%%%%%%%%%%%%%%%%%%%%%%%%%
%%% GENERAL DOCUMENTATION BEGINS 
%%%%%%%%%%%%%%%%%%%%%%%%%%%%%%%%%%%%%%%%%%%%%%%%%%%%%%%%%%%%%%%%%%%%%%%%
\newpage
\setcounter{section}{5}
\section{}
Show that 
\begin{align}
    \left( \begin{array}{c|c|c} 
            I_k & 0 & 0 \\ \hline 0 & 1 & 0 \\ \hline 0 & -l_{21} & I
        \end{array} \right)^{-1}
    =
    \left( \begin{array}{c|c|c} 
            I_k & 0 & 0 \\ \hline 0 & 1 & 0 \\ \hline 0 & l_{21} & I
        \end{array} \right)
\end{align}

\begin{proof}
    To show the inverse relation between two matrices, we multiply two matrices 
    \begin{align}
        \left( \begin{array}{c|c|c} 
            I_k & 0 & 0 \\ \hline 0 & 1 & 0 \\ \hline 0 & -l_{21} & I
        \end{array} \right)
    \left( \begin{array}{c|c|c} 
            I_k & 0 & 0 \\ \hline 0 & 1 & 0 \\ \hline 0 & l_{21} & I
        \end{array} \right)
    = 
    \left( \begin{array}{c|c|c} 
            I_k & 0 & 0 \\ \hline 0 & 1 & 0 \\ \hline 0 & -l_{21} + l_{21} & I 
        \end{array} \right)
    = I
    \end{align}
    Then it is proved that 
\begin{align}
    \left( \begin{array}{c|c|c} 
            I_k & 0 & 0 \\ \hline 0 & 1 & 0 \\ \hline 0 & -l_{21} & I
        \end{array} \right)^{-1}
    =
    \left( \begin{array}{c|c|c} 
            I_k & 0 & 0 \\ \hline 0 & 1 & 0 \\ \hline 0 & l_{21} & I
        \end{array} \right)
\end{align}
\end{proof}
%%%%%%%%%%%%%%%%%%%%%%%%%%%%%%%%%%%%%%%%%%%%%%%%%%%%%%%%%%%%%%%%%%%%%%%%
\setcounter{section}{6}
\newcommand{\Lt}{\tilde{L}}
\section{}
Assume that 
$ \Lt_{k-1} = 
       \left( \begin{array}{c|c|c} 
               L_{00} & 0 & 0 \\ \hline 
               l_{10}^T & 1 & 0 \\ \hline
               L_{20} & 0 & I
        \end{array} \right)
$
, show that 
$ \Lt_{k} = 
       \left( \begin{array}{c|c|c} 
               L_{00} & 0 & 0 \\ \hline 
               l_{10}^T & 1 & 0 \\ \hline
               L_{20} & l_{21} & I
        \end{array} \right)
$.

\begin{proof}
    It is known that 
    \begin{align}
        L_k = \hat{L}_{k}^{-1} = 
       \left( \begin{array}{c|c|c} 
               I & 0 & 0 \\ \hline 
               0 & 1 & 0 \\ \hline
               0 & -l_{21} & I
           \end{array} \right)^{-1}
       =
       \left( \begin{array}{c|c|c} 
               I & 0 & 0 \\ \hline 
               0 & 1 & 0 \\ \hline
               0 & l_{21} & I
           \end{array} \right)
    \end{align}
    Then we have
    \begin{align} 
        \Lt_k = \Lt_{k-1} L_k  
        =  
        \left( \begin{array}{c|c|c} 
               L_{00} & 0 & 0 \\ \hline 
               l_{10}^T & 1 & 0 \\ \hline
               L_{20} & 0 & I
        \end{array} \right)
        \left( \begin{array}{c|c|c} 
               I & 0 & 0 \\ \hline 
               0 & 1 & 0 \\ \hline
               0 & l_{21} & I
           \end{array} \right)
       =  
       \left( \begin{array}{c|c|c} 
               L_{00} & 0 & 0 \\ \hline 
               l_{10}^T & 1 & 0 \\ \hline
               L_{20} & l_{21} & I
        \end{array} \right)
    \end{align}
    which tells us that $\Lt_k$ can be computed by simply placing computed
    vector $l_{21}$ below the diagonal of a unit lower triangular matrix.
\end{proof}
%%%%%%%%%%%%%%%%%%%%%%%%%%%%%%%%%%%%%%%%%%%%%%%%%%%%%%%%%%%%%%%%%%%%%%%%
\newpage
\setcounter{section}{11}
\section{}
\subsection{Variant 1: Overwrite $A$ and $L$}
\lstinputlisting{../FLAME@lab/LU_pp_unb_var1.m}

\subsection{Variant 2: Overwrite $A$}
\lstinputlisting{../FLAME@lab/LU_pp_unb_var2.m}

%%%%%%%%%%%%%%%%%%%%%%%%%%%%%%%%%%%%%%%%%%%%%%%%%%%%%%%%%%%%%%%%%%%%%%%%
\newpage
\setcounter{section}{12}
\section{}
{\bf Derivation}
Partition 
$U=\left(\begin{array}{c|c} v_{11}& u_{12}^T \\\hline 0 & U_{22} \end{array}\right)$,
$x = \left( \begin{array}{c} \chi_1 \\\hline x_2 \end{array} \right)$ and 
$y = \left( \begin{array}{c} \psi_1 \\\hline y_2 \end{array} \right)$. Then
\begin{align}
    Ux = 
    U=\left(\begin{array}{c|c} v_{11}& u_{12}^T \\\hline 0 & U_{22} \end{array}\right)
    \left( \begin{array}{c} \chi_1 \\\hline x_2 \end{array} \right) 
    = 
    \left( \begin{array}{c} v_{11}x_1 + u_{12}^T x_2 \\\hline U_{22} x_2 \end{array} \right) 
    =
    \left( \begin{array}{c} \psi_1 \\\hline y_2 \end{array} \right)
\end{align}
Then the update is given by
\begin{align}
    \chi_1 = \frac{1}{v_{11}}(\psi_1 - u_{12}^T x_2)
\end{align}


\noindent
{\bf Implementation}
\lstinputlisting{../FLAME@lab/UTS_Solver_unb_var1.m}

%%%%%%%%%%%%%%%%%%%%%%%%%%%%%%%%%%%%%%%%%%%%%%%%%%%%%%%%%%%%%%%%%%%%%%%%
\newpage
\setcounter{section}{13}
\section{}
{\bf Derivation}.
Partition $U=\left(\begin{array}{c|c} U_{11}& u_{12} \\\hline 0 & v_{22} \end{array}\right)$,
$x = \left( \begin{array}{c} x_1 \\\hline \chi_2 \end{array} \right)$ and 
$y = \left( \begin{array}{c} y_1 \\\hline \psi_2 \end{array} \right)$. Then
\begin{align}
    Ux = 
    \left(\begin{array}{c|c} U_{11}& u_{12} \\\hline 0 & v_{22} \end{array}\right)
    \left( \begin{array}{c} x_1 \\\hline \chi_2 \end{array} \right) 
    = 
    \left( \begin{array}{c} U_{11}x_1 + \chi_2 u_{12} \\\hline \chi_2 v_{22} \end{array} \right) 
    =
    \left( \begin{array}{c} y_1 \\\hline \psi_2 \end{array} \right)
\end{align}
Then the update is given by
\begin{align}
    \chi_2 &= \frac{\psi_2}{v_{22}} \\
    U_{11} x_1 &= y_1 - \chi_2 u_{12} = y_1 - \frac{\psi_2}{v_{22}} u_{12}
\end{align}

\noindent
{\bf Implementation}.
\lstinputlisting{../FLAME@lab/UTS_Solver_unb_var2.m}

%%%%%%%%%%%%%%%%%%%%%%%%%%%%%%%%%%%%%%%%%%%%%%%%%%%%%%%%%%%%%%%%%%%%%%%%
%%% General Documentation ends
%%%%%%%%%%%%%%%%%%%%%%%%%%%%%%%%%%%%%%%%%%%%%%%%%%%%%%%%%%%%%%%%%%%%%%%%
\end{document}
