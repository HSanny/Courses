%%%%%%%%%%%%%%%%%%%%%%%%%%%%%%%%%%%%%%%%%%%%%%%%%%%%%%%%%%%%%%%%%%%%%%%%
%%%  THIS TEX FILE IS TO GENERATE PDF FILE FOR 
%%% 
%%%  COPYRIGHT (C) JIMMY LIN, 2013, UT AUSTIN
%%%%%%%%%%%%%%%%%%%%%%%%%%%%%%%%%%%%%%%%%%%%%%%%%%%%%%%%%%%%%%%%%%%%%%%%
\documentclass[11pt,a4paper]{article}
%%%%%%%%%%%%%%%%%%%%%%%%%%%%%%%%%%%%%%%%%%%%%%%%%%%%%%%%%%%%%%%%%%%%%%%%
%%%  PACKAGES USED IN THIS TEX SOURCE FILE
%%%%%%%%%%%%%%%%%%%%%%%%%%%%%%%%%%%%%%%%%%%%%%%%%%%%%%%%%%%%%%%%%%%%%%%%
\usepackage{geometry,amsthm,amsmath,graphicx,fancyheadings,fancybox}
\usepackage[colorlinks,
            linkcolor=blue,
            anchorcolor=red,
            citecolor=green
            ]{hyperref}
\usepackage{/Users/JimmyLin/workspace/latexTemplate/UTA_CS/JS}
\usepackage{/Users/JimmyLin/workspace/latexTemplate/UTA_CS/JSASGN}
%%%%%%%%%%%%%%%%%%%%%%%%%%%%%%%%%%%%%%%%%%%%%%%%%%%%%%%%%%%%%%%%%%%%%%%%
%%% MACROS CONTAINING THE FILE INFORMATION
%%%%%%%%%%%%%%%%%%%%%%%%%%%%%%%%%%%%%%%%%%%%%%%%%%%%%%%%%%%%%%%%%%%%%%%%
\renewcommand{\COURSE}{CS331 Algorithm}
\renewcommand{\LECTURER}{Greg Plexton}
\renewcommand{\TUTOR}{Chunzhi Zhu}
\renewcommand{\TASK}{Assignment 03}
\renewcommand{\RELEASEDATE}{February. 19 2014}
\renewcommand{\DUEDATE}{February. 26 2014}
\renewcommand{\TIMECONSUME}{10 hours}
%%%%%%%%%%%%%%%%%%%%%%%%%%%%%%%%%%%%%%%%%%%%%%%%%%%%%%%%%%%%%%%%%%%%%%%%
%%% DOCUMENTATION STARTS FROM HERE 
%%%%%%%%%%%%%%%%%%%%%%%%%%%%%%%%%%%%%%%%%%%%%%%%%%%%%%%%%%%%%%%%%%%%%%%%
\begin{document}
%%%%%%%%%%%%%%%%%%%%%%%%%%%%%%%%%%%%%%%%%%%%%%%%%%%%%%%%%%%%%%%%%%%%%%%%
%% TITLE PAGE
%%%%%%%%%%%%%%%%%%%%%%%%%%%%%%%%%%%%%%%%%%%%%%%%%%%%%%%%%%%%%%%%%%%%%%%%
\begin{titlepage}
    \maketitle
\end{titlepage}
%%%%%%%%%%%%%%%%%%%%%%%%%%%%%%%%%%%%%%%%%%%%%%%%%%%%%%%%%%%%%%%%%%%%%%%%
%% CONTENT PAGE: TABLEOFCONTENTS, LISTOFTABLES, LIST OF FIGURES
%%%%%%%%%%%%%%%%%%%%%%%%%%%%%%%%%%%%%%%%%%%%%%%%%%%%%%%%%%%%%%%%%%%%%%%%
\renewcommand{\contentsname}{Contents}
\begin{center} 
    \tableofcontents 
\end{center}
\newpage
%%%%%%%%%%%%%%%%%%%%%%%%%%%%%%%%%%%%%%%%%%%%%%%%%%%%%%%%%%%%%%%%%%%%%%%%
%%% GENERAL DOCUMENTATION BEGINS 
%%%%%%%%%%%%%%%%%%%%%%%%%%%%%%%%%%%%%%%%%%%%%%%%%%%%%%%%%%%%%%%%%%%%%%%%
\newcommand{\wm}[1]{\ensuremath{w(M_{#1})}}
\newcommand{\wmp}[1]{\ensuremath{w(M'_{#1})}}
\newcommand{\wmi}{\ensuremath{w(M_i)}}
\newcommand{\wmip}{\ensuremath{w(M'_i)}}
\newcommand{\constantTime}{\ensuremath{\mathcal{O}(1)}}
\newcommand{\linearTime}{\ensuremath{\mathcal{O}(n)}}
\newcommand{\lbw}[1]{\ensuremath{a_{#1} q_{#1} + b_{#1}}}  %% linear bid weight
\newcommand{\sbw}[1]{\ensuremath{m_{#1}}}
\section{Exercise 3}
\ovalbox{
\begin{minipage}{15cm}
Show how to compute $\wmi$ for all $i$ such that $0 \leq i < |A|$ in a total of
$O(n)$ time.
\end{minipage} }

\subsection{Overall Idea}
Since there are $n$ items, if we want to compute all the $w(M_i)$, the
technique of dynamic programming is first one to consider. Specifically, we
can compute $\wm{0}$, cache it and then compute $\wm{1}$ based on $\wm{0}$ by
constant time $\constantTime$. And then we repeat this process for $n$ times.
In this manner, we derive all $n$ $\wmi$ in $\linearTime$.\\
In the case, each $\wm{i}$ represents one matching with one linear bid
removed under configuration $U''$. 

\subsection{Mathematical Notation}
Before introducing procedural description and mathematical proof, we first
present a series of mathematical notations as follows: 
\begin{itemize}
    \item{$\wmi$ is the weight of MWMCM $\wmi$.}
    \item{$a_l$ slope of one linear bid with index $l$.}
    \item{$b_l$ intercept of one linear bid with index $l$.}
    \item{$q_l$ quality of one item with index $l$.}
    \item{$m_l$ offer of one bid with index $l$.}
\end{itemize}

Besides, all linear bids are sorted in descending order according to the
slope. And there is no need to sort the single-item bids in this scenario. 

\subsection{Algorithmic Description}
Given the sorted array of linear bid in terms of its slope and sorted array of
single item bid in terms of its offer price. 

\begin{itemize}
    \item{First of all, we compute the $\wm{0}$ in $\linearTime$.
        Specifically: }
        \begin{itemize}
        \item{Match all single-item bids to items according to the provided
            one-one correspondence}
        \item{Remove all matched item from the sorted array of item}
        \item{Match all linear bids to the remained items by virtue of
                matching the linear bid with the highest slope to the item
            with the highest quality}
        \end{itemize}

    \item{Then, we sequentially compute $\wm{k}$ based on $\wm{k-1}$ at
        $\constantTime$. The formula for incremental computation is given as
    follows: }
    $$
    \wm{k} = \wm{k-1} + (a_{k-1} q_{k-1} + b_{k-1}) 
        - (a_k q_k + b_k)
    $$
\end{itemize}
 
Since there are one linear time computation and $k-1$ constant time
computation. The time complexity of this method is in linear time
$\linearTime$.

\newpage
\subsection{Mathematical Proof}
Since the computation of $\wm{0}$ as MWMCM in $mcm(G', U''-A[0])$ is justified
in the last assignment, you can reference my solution in the assignment 2 for
the proof of that part. And in the following, what I am doing is to validate
the formula for incremental computation. 

\begin{align*}
    \wm{0} = \Sigma_{n \neq 0} (a_n q_n + b_n) + \Sigma_s m_s \\
    \wm{1} = \Sigma_{n \neq 1} (a_n q_n + b_n) + \Sigma_s m_s 
\end{align*}
$$
\vdots 
$$
\begin{align*}
    \wm{k} = \Sigma_{n \neq k} (a_n q_n + b_n) + \Sigma_s m_s
\end{align*}

We observe that the incremental formula is 
    $$
    \wm{k} = \wm{k-1} + (\lbw{k-1}) - (\lbw{k})
    $$

Then, we prove it by mathematical induction. \\[0.5cm]
\textbf{Base case}: 

for $k = 0$, obviously it does hold.
\begin{align*}
    \wm{1} &= \Sigma_{n \neq 1} (\lbw{n}) + \Sigma_s m_s \\
           &= \Sigma_{n \neq 1} (\lbw{n}) + \Sigma_s m_s 
    + (\lbw{1}) - (\lbw{0}) - (\lbw{1}) + (\lbw{0}) \\
    &= \Sigma_{n}(\lbw{n}) + \Sigma_s m_s - (\lbw{0}) - (\lbw{1}) + (\lbw{0})  \\
    &= \Sigma_{n \neq 0} (\lbw{n}) + \Sigma_s m_s + (\lbw{0}) - (\lbw{1})  \\
    &= \wm{0} + (\lbw{0}) - (\lbw{1}) 
\end{align*}
\textbf{Inductive cases}: 

Assume that the relationship between the case of
$k-1$ and $k$ holds, now we prove that it also holds for $k$ and $k+1$. ($k+1$
items in total).  

That is to say, given
\begin{align*}
    \wm{k} = \wm{k-1} + (\lbw{k-1}) - (\lbw{k})
\end{align*}

We need to prove
\begin{align*}
    \wm{k+1} = \wm{k} + (\lbw{k}) - (\lbw{k+1})
\end{align*}

By the definition of $\wm{k+1}$ and $\wm{k}$, it can be easily derived that
the inductive cases hold. The specific proof procedure is similar to the base
case. 

\newpage
\section{Exercise 4}
\ovalbox{
\begin{minipage}{15cm}
Show how to compute $w(M'_i)$ for all $i$ such that $0 \leq i < |B|$ in a total of
$O(n)$ time.
\end{minipage}}
\subsection{Overall Idea}
This case is slightly more complicated than the previously introduced
scenario. This is because the linear bids are the source of instability.
However, we can use the same approach as that of solving Exercise 3. Briefly,
the base computation for $\wmip$ remained unvaried, whereas the following
incremental computation became more complex.

\subsection{Mathematical Notation}
Everything used in the solution for exercise 3 is still available in this
section except that the single-item array should also be sorted in terms of
the quality of item it offers to. \\
That is, for all single-item bids $s$
\begin{align*} 
    \text{single[0].offerto.quality} \geq \text{single[1].offerto.quality}
    \geq \dots \geq \text{single[s].offerto.quality}
\end{align*}
Additionally, we have to conceptualize new concepts as follows:
\begin{itemize}
    \item{$a'_s$ represents the slope of linear bid matched to the item, which
        is originally matched to single-item bid $s$.}
    \item{$b'_s$ represents the intercept of linear bid matched to the item,
            is originally matched to single-item bid $s$.}
    \item{$q_s$ represents the quality of item which
        single-item bid $s$ offers to.}
\end{itemize}

\subsection{Algorithmic Description}
Given the sorted array of linear bids in terms of its slope and sorted array of
single-item bids in terms of the \textbf{quality of item it offers to}. 

\begin{itemize}
    \item{First of all, we compute the $\wmp{0}$ in $\linearTime$.
        Specifically: }
        \begin{itemize}
            \item{Match all single-item bids within $U''-u$ to items according to
                the provided one-one correspondence.}
            \item{Remove all matched items from the sorted array of items.}
            \item{Match all linear bids to the remained items by virtue of
                    matching the linear bid with the highest slope to the item
                with the highest quality.}
        \end{itemize}
    \item{Then, we compute $\wmp{m}$ based on $\wmp{m-1}$ at $\constantTime$}
        \begin{itemize}
        \item{Compute the update on matching made by single-item matching,
                that is to add the offer price of single-item bid indexed by
                $s-1$ back and subtract the offer price of single-item bid
            indexed by $s$.}
        \item{Compute the update on matching made by linear bid. It is a
            little more complicated since the change of single-item bid will
        deprive the original matching made by linear bids. We rematch all the
        items whose quality is between the quality of the item we added
        and that of the item we subtracted.} 
    \end{itemize}
    \item{To summarize, at the step above the mathematical formula for
        incremental computation is given as follows: } $$ 
    \wmp{m} = \wmp{m-1} + U_{single} + U_{linear}
    $$ 
    where $U_{single}$ corresponds to update on single-item bid matching,
    $$ 
    U_{single} = \sbw{s-1} - \sbw{s}
    $$ 
    and $U_{linear}$ corresponds to update on linear bid matching,
    $$ 
    U_{linear} = \Sigma_{q_{s-1} \leq q_j \leq q_{s}} (q_j * (a'_{j-1} - a'_j)) 
    $$ 
\end{itemize}
 
Since there are one linear time computation and $m-1$ constant time
computation. Precisely speaking, a particular incremental computation for each
$\wmp{m}$ is not constant but all the whole stage of incremental computing is
in linear time. Thus, The time complexity of this method is in linear time
$\linearTime$.

%%%%%%%%%%%%%%%%%%%%%%%%%%%%%%%%%%%%%%%%%%%%%%%%%%%%%%%%%%%%%%%%%%%%%%%%
%%% General Documentation ends
%%%%%%%%%%%%%%%%%%%%%%%%%%%%%%%%%%%%%%%%%%%%%%%%%%%%%%%%%%%%%%%%%%%%%%%%
\end{document}
