%%%%%%%%%%%%%%%%%%%%%%%%%%%%%%%%%%%%%%%%%%%%%%%%%%%%%%%%%%%%%%%%%%%%%%%%
%%%  THIS TEX FILE IS TO GENERATE PDF FILE FOR 
%%% 
%%%  COPYRIGHT (C) JIMMY LIN, 2013, UT AUSTIN
%%%%%%%%%%%%%%%%%%%%%%%%%%%%%%%%%%%%%%%%%%%%%%%%%%%%%%%%%%%%%%%%%%%%%%%%
\documentclass[11pt,a4paper]{article}
%%%%%%%%%%%%%%%%%%%%%%%%%%%%%%%%%%%%%%%%%%%%%%%%%%%%%%%%%%%%%%%%%%%%%%%%
%%%  PACKAGES USED IN THIS TEX SOURCE FILE
%%%%%%%%%%%%%%%%%%%%%%%%%%%%%%%%%%%%%%%%%%%%%%%%%%%%%%%%%%%%%%%%%%%%%%%%
\usepackage{geometry,amsthm,amsmath,graphicx,fancyheadings}
\usepackage[colorlinks,
            linkcolor=blue,
            anchorcolor=red,
            citecolor=green
            ]{hyperref}
\usepackage{/Users/JimmyLin/workspace/latexTemplate/UTA_CS/JS}
\usepackage{/Users/JimmyLin/workspace/latexTemplate/UTA_CS/JSASGN}
%%%%%%%%%%%%%%%%%%%%%%%%%%%%%%%%%%%%%%%%%%%%%%%%%%%%%%%%%%%%%%%%%%%%%%%%
%%% MACROS CONTAINING THE FILE INFORMATION
%%%%%%%%%%%%%%%%%%%%%%%%%%%%%%%%%%%%%%%%%%%%%%%%%%%%%%%%%%%%%%%%%%%%%%%%
\renewcommand{\COURSE}{CS371D Distributed System}
\renewcommand{\LECTURER}{Lorenzo Alvisi}
\renewcommand{\TUTOR}{Chao Xie}
\renewcommand{\TASK}{Problem Set 01}
\renewcommand{\RELEASEDATE}{Feb. 8 2014}
\renewcommand{\DUEDATE}{Feb. 12 2014}
\renewcommand{\TIMECONSUME}{20 hours}
%%%%%%%%%%%%%%%%%%%%%%%%%%%%%%%%%%%%%%%%%%%%%%%%%%%%%%%%%%%%%%%%%%%%%%%%
%%% DOCUMENTATION STARTS FROM HERE 
%%%%%%%%%%%%%%%%%%%%%%%%%%%%%%%%%%%%%%%%%%%%%%%%%%%%%%%%%%%%%%%%%%%%%%%%
\begin{document}
%%%%%%%%%%%%%%%%%%%%%%%%%%%%%%%%%%%%%%%%%%%%%%%%%%%%%%%%%%%%%%%%%%%%%%%%
%% TITLE PAGE
%%%%%%%%%%%%%%%%%%%%%%%%%%%%%%%%%%%%%%%%%%%%%%%%%%%%%%%%%%%%%%%%%%%%%%%%
\begin{titlepage}
    \maketitle
\end{titlepage}
%%%%%%%%%%%%%%%%%%%%%%%%%%%%%%%%%%%%%%%%%%%%%%%%%%%%%%%%%%%%%%%%%%%%%%%%
%% CONTENT PAGE: TABLEOFCONTENTS, LISTOFTABLES, LISTOFFIGURES
%%%%%%%%%%%%%%%%%%%%%%%%%%%%%%%%%%%%%%%%%%%%%%%%%%%%%%%%%%%%%%%%%%%%%%%%
\renewcommand{\contentsname}{Contents}
\begin{center} 
    \tableofcontents 
    %\listoftables 
    %\listoffigures
\end{center}
\newpage
%%%%%%%%%%%%%%%%%%%%%%%%%%%%%%%%%%%%%%%%%%%%%%%%%%%%%%%%%%%%%%%%%%%%%%%%
%%% GENERAL DOCUMENTATION BEGINS 
%%%%%%%%%%%%%%%%%%%%%%%%%%%%%%%%%%%%%%%%%%%%%%%%%%%%%%%%%%%%%%%%%%%%%%%%

\section{Problem 1: Snapshot Protocol}
To relax the FIFO assumption under the asynchronous system, the problems
encountered are as follows: 
\begin{itemize}
    \item{Messages in channel will reach destination after the empty message
            or "take a snapshot" is delivered. In this case, we should delay the
            report of one process to the snapshot initiator, so as to reserve
        enough time waiting for the arrival of messages in channel. }
    \item{Messages sent after the delivery of empty message or "take a
            snapshot" message may arrive at destination process prior to it.
            In this case, some messages that are monitored should not be
            counted as component of channel state.} 
\end{itemize}

Here we choose process $p_0$ as a snapshot initiator.
\begin{itemize}
    \item{processor $p_0$ selects Ω}
    \item{$p_0$ sends “take a snapshot at Ω” to all processes; it waits for all
            of them to reply and then sets its logical clock to Ω}
    \item{when clock of $p_i$ read Ω}
        \begin{itemize}
            \item{records its local state $\sigma_i$}
            \item{sends an empty message with timestamp $TS(m)$ along its outgoing channels}
            \item{starts recording messages received on each incoming channel}
            \item{stops recording a channel when $p_i$ read delay time $\tau$}
        \end{itemize}
    \item{when $p_i$ receive empty messages with timestamp $TS(m)$ from $p_j$}
        \begin{itemize}
            \item{ignore all received messages with timestamp greater than
                    $TS(m)$. That is, do not report these messages as part of
                    channel state.}
        \end{itemize}
    \item{when $p_i$ has received empty messages from all incoming channels
            and reach the delay time $\tau$, it sends collected state to $p_0$ and
            stops.} 
\end{itemize}

Now let us prove the approach above generate the consistent global state.

\textbf{Proof}: 

\newpage
\section{Problem 2: Property Specification}
Before providing solutions for these statement, we need first clarify the
semantics of safety and liveness property 

(1)\textbf{Liveness}. Even liveness property does not hold now, it will hold
in one future state.

(2)\textbf{Safety}. Once saftey property holds, that property holds for all
prefix state.
\\
Solutions: 

1. \textbf{Safety}. 
Once we know that another general will follow the attack decision within five
minutes, such decision delay will hold in all previous global state. Thus,
this is safety property. This is not liveness property because once another
general follows the attack decision over 5 minutes, this property would never
hold in the future.

2. \textbf{Safety}. If the statement that no general decide to attack holds, no
general can decide to attack in the previous state. Hence, this is safety
property. But this is not liveness property because once there is one general
decides to attack, this statement cannot hold in any future state.

3. \textbf{A combination of safety and liveness}. If both generals will eventually
decide on the same value, such tendency also holds for all previous states.
Therefore, this is safety property. This is also liveness property because
even though two generals do not tend to reach the same decision in one state,
they may tend to reach the same decision since new decision is made and
communication does not fail.

4. \textbf{No property at all}. The statement of "Not to attack is a possible
decision" does not satisfy any property. Possibility of not attacking at the
moment does not indicate the same possibility in the previous state.
Impossibility of not attacking does not indicate possibility of not attacking
in the future state.

5. \textbf{Safety}. This is safety property because one general in previous
states can only have less number of messages. No more than 10 messages now
indicates no more than 10 messages in the previous states. This is not
liveness property because the general may send more than 10 messages in the
future.

6. \textbf{Liveness}. This statement satisfy liveness property because the
even the general B does send 5 messages now, it may sends at least 5 messages
in the future. This is not safety property because general will send less
messages in the past. Thus, the lower bound of number of messages may not be
satisfied in previous states.

\newpage
\section{Problem 3}
\subsection{Under Bounded Loss Assumption}
The algorithm is as follows: 
\begin{itemize} 
    \item{If one general $g$ are ready to attack, and he does not receive "Yes"
        message from the other general before: }
        \begin{itemize}
            \item{General $g$ should send eleven "Yes" messages to the other general}
            \item{General $g$ decides to attack once it receives "Yes" message
                    from the other general}
            \item{General $g$ decides "not to attack" if it receives "No" message
                    from the other general}
        \end{itemize}
    \item{If one general $g$ are ready to attack, and he did receive "Yes"
        message from the other general before: }
        \begin{itemize}
            \item{$g$ should send eleven "Yes" messages to the other general.
                    (acknowledgement)}
            \item{General $g$ make up its decision to "attack".}
        \end{itemize}
    \item{If one general $g$ are not ready to attack, and did not receive one "Yes"
            message from the other general}
        \begin{itemize}
            \item{General $g$ make up its decision "not to attack".}
            \item{If received "Yes" message from the other general, $g$ should
                    send eleven "No" messages to the other general}
            \item{If received "No" message from the other general, $g$ should
                    send eleven "No" messages to the other general as
                    acknowledgement. }
        \end{itemize}
\end{itemize}

\textbf{Proof for Agreement}. 

If general A is ready to attack, it should send 11 "Yes" messages to
the other general. And general B must receive at least one message
according to the assumption that at most 10 messages are lossy. Then if
general B is ready, he should also send more than 10 "Yes" messages to general A.
In this manner, general A is guaranteed to receive general B's decision. And
both general decides to attack. (vice versa)

In all other cases, general A and general B will both decide not to attack
based on the algorithm shown above.

\textbf{Proof for Validity}. 

1. If both generals are not ready, each general will decide "not to attack"
by himself without communicating with each other. In this case, no one decides
to attack.

2. If both generals are ready, both of them will send 11 "Yes" messages to the other
general. And it is guaranteed that each general will at least receive one
"Yes" message. And since both of them are ready and did not send "No" messages
to the other, neither generals will decide "not to attack" for sure. 

In summary, since the point 1 and point 2 hold, the Validity Property
holds for the provided algorithm.

\textbf{Proof for Termination}. 

If one general are not ready, it will decide "not to attack" at the moment. 

If one general are ready, it will send 11 "Yes" message and these messages
will be received for sure. The receiver general will respond with 11 messages.
If these messages are "Yes" , the sender general will decide to "attack",
otherwise, it will decides "not to attack".  Thus, since every general
eventually make decision as shown above, the termination property for this
algorithm holds. 

\newpage
\subsection{Under Eventual Delivery Assumption}
The difference of this question lies in the fact that each general has no
prior knowledge about the number of messages that would be lost. We can
first let two generals send redundant messages to fail the channel until it is
reliable. 

The algorithms designed are as follows:
\begin{itemize}
    \item{If one general is not ready to attack}
        \begin{itemize}
        \item{keeps on sending "No" message to the other general}
        \item{decide "not to attack" no matter what it received}
        \end{itemize}
    \item{If one general is ready to attack}
        \begin{itemize}
        \item{keeps on sending "Yes" to the other general}
        \item{decide "not to attack" if "No" message is received}
        \item{decide "attack" if "Yes" message is received}
        \end{itemize}
\end{itemize}

\textbf{Proof for Agreement}. 

If both general A and general B are ready to attack, it will continuously send
"Yes" messages to the other general. And general B must eventually receive at
least one "Yes" message according to the eventual delivery assumption. General B
follows the same analysis. Since general A and B will eventually receive "Yes"
message and are both ready to attack, they will both decide to "attack".
Therefore, both generals decide the same value if both are ready.

According to the algorithm, if one of two generals (say, general A) is not
ready to attack, he will keeps on sending "No" messages to the other general
(general B) and he will decide "not to attack". Even if general (B) is ready
to attack, since it receives "No" message, it will also decide "not to
attack". Hence, two generals decide the same value if either one is not ready.

If both generals are not ready to attack, they will stick to sending "No"
message and directly decide "not to attack". Thus, two generals decide the
same value if none of them is ready.

In summary, the given algorithm satisfies Agreement Property.

\textbf{Proof for Validity}. 

1. If both generals are not ready, they will directly decide "not to attack"
and telling the other general their decision. Hence, in this case, no general
would decide "attack".

2. If both generals are ready, they both keep on sending "Yes" message to the
other general and eventually they will receive "Yes" message sent by the other
general. After the recept of "Yes" message, since they are both ready, neither
of them will decide "not to attack".

In summary, since the point 1 and point 2 hold, the Validity Property
holds for the provided algorithm.

\textbf{Proof for Termination}. 

Apparently, the Termination Property holds for the algorithm. This is because
no matter whether one general is ready or not, he will keep on sending message
to the other general. The other general will eventually receive one of those
message and make decision in terms of the recept message and its own readiness.

\newpage
\section{Problem 4}
\textbf{Proof}:
$$ VC (\Sigma_{min}^{\sigma_i^k}) = VC (\sigma_i^k)$$

By defintion, 
$ \Sigma_{min}^{\sigma_{i}^{k}}$ is the earliest consistent state that  $
\sigma_i^k$ (event $k$ at process $i$) can belong to. Thus, we divide the
entire proof into two parts. For the first part we show that
$VC(\sigma_{i}^{k})$ represents a consistent global state. On the second part , we 
show that each component in $VC(\sigma_{i}^{k})$ is earliest.

\subsection{$VC(\sigma_i^k)$ is earliest}
First of all, we prove that 
    \textbf{the global state represented by $VC(\sigma_i^k)$ is consistent}.
    Actually, we can refer to, for convenience, the conclusion made in class
    note that a vector clock of one event correspond to a consistent global
    state. But we still provide detailed proof by contradiction.

    Assume that the global state represented by $VC(\sigma_i^k)$ is
    \textbf{not} consistent. That is to say,

    \begin{align}
     \exists \sigma_j^{s^*}, \sigma_j^{s^*} \rightarrow \sigma_i^k, s.t. 
    \end{align}
    \begin{align}
     \sigma_i^k \in C, \sigma_j^{s*} \notin C 
    \end{align}

    According to the VC property, the happen-before relation above indicates
    $$ VC(\sigma_j^{s^*}) < VC(\sigma_i^{k})$$

    That is,
    \begin{align}
    VC(\sigma_j^{s^*}) \neq VC(\sigma_i^{k}) \ and \ \forall x,
    VC(\sigma_j^{s^*})[x] \leq VC(\sigma_i^{k})[x]  \label{instan}
    \end{align}

    However, if we take an event at process $j$ happening before
    $\sigma_j^{s*}$ and belong to the consistent cut, let's call it
    $\sigma_j^s$. According the update rule of vector clock, we have
    \begin{align}
    VC(\sigma_j^s)[j] < VC(\sigma_j^{s^*})[j]
    \end{align}

    Instantiate the formula \eqref{instan}, we have
    \begin{align}
        (\sigma_j^{s^*})[j] \leq VC(\sigma_i^{k})[j]
    \end{align}

    Hence, we have 
    \begin{align} \label{vio1}
    VC(\sigma_j^s)[j] < VC(\sigma_j^{s^*})[j] \leq VC(\sigma_i^{k})[j]
    \end{align}

    However, according to the property of consistent cut (see the class note),
    the event $\sigma_j^s$ and $\sigma_i^{k}$ are in one consistent cut and
    should satisfy
    \begin{align} \label{vio2}
     VC (\sigma_j^s)[j] \geq VC (\sigma_i^{k})[j] 
    \end{align}

    Obviously, the conflict lies in \eqref{vio1} and \eqref{vio2}. Hence, we
    should negate the assumption proposed at the very beginning and conclude
    that \textbf{the global state represented by $VC(\sigma_i^k)$ is
        consistent}.

It is obvious that this consistency property holds (refer to the class notes),
out of the simple strong clock condition and concurrency property of Vector
Clock. And since in the global state, the vector clock is computed according
to the happen-before relationship on both internal event and external event,
$\Sigma_{min}(\sigma_i^k)$ is consistent.

\newpage
\subsection{$VC(\sigma_i^k)$ is earliest}
Then, we prove that 
    \textbf{for arbitrary $\sigma_i^k$, those each local state in
        $VC(\sigma_i^k)$ is earliest among all possible ones consistent with
        $\sigma_i^k$}.

    We prove this by mathematical induction on number of events $E$. 

    \textbf{For base case of $E = 1$}, there is only one event (say, at process $i$), the vector
    clock of that event 
        
    \begin{eqnarray}
        \forall j, 1 \leq j \leq n,\ VC(\sigma_i^1)[j] =
        \begin{cases}
        0                   & j \neq i \\
        1                   & j = i
        \end{cases}
    \end{eqnarray}
    
    Obviously, the claim to be proven is true for the case of $E = 1$ since all local
    states are earliest.

    \textbf{For base case of $E = 2$}, we consider two possible derivation according to the update
    rule of vector clock. One is local update rule, and the other is
    happen-before update rule. And third possibility is one independent event,
    regardless of the update rule.

    \begin{itemize}

        \item{For local update rule, we have: }
    \begin{eqnarray}
       \forall j, 1 \leq j \leq n,\ VC(\sigma_i^2)[j] = 
        \begin{cases}
            0   & j \neq i \\
            2   & j = i
        \end{cases}
    \end{eqnarray}

    Apparently, this vector clock is earliest.
    
\item{For happen-before update rule, we introduce a event $e_k^1$ at process
        $k$, ($k \neq i,\ 1 \leq k \leq n$) s.t. $e_i^1
        \rightarrow e_k^1$, and have: }
    \begin{eqnarray}
       \forall j, 1 \leq j \leq n,\  VC(\sigma_k^1)[j] = 
        \begin{cases}
            1   & j = i\ or\ j = k \\
            0   & otherwise
        \end{cases}
    \end{eqnarray}

    By observation, this vector clock is earliest as well.

\item{For introducing an independent event, it is obvious that the vector clock
        of new event is still earliest. Here we omit the detail for convenience.}
    \end{itemize}

    Now we can conclude that the claim to be proven is true for $E = 2$.

    \textbf{Generalization cases}. We assume that for $E = k$, all components
    in $VC(\sigma_i^k)$ is earliest for any $\sigma_i^k$. And prove that for
    $E = k + 1$, the same property holds. Similarly, we only needs to consider
    newly introduced event. The detailed proof follows the generalization from
    $E = 1$ to $E = 2$, involving the case of two VC update rule and one
    independent case. 
    
\subsection{Sum Up}
Since we have proven that the global state corresponding to $VC(\sigma_i^k)$
is consistent and also earliest reachable state, it is reasonable to conclude
that 
\begin{align}
 VC (\Sigma_{min}^{\sigma_i^k}) = VC (\sigma_i^k)
\end{align}

%%%%%%%%%%%%%%%%%%%%%%%%%%%%%%%%%%%%%%%%%%%%%%%%%%%%%%%%%%%%%%%%%%%%%%%%
%%% General Documentation ends
%%%%%%%%%%%%%%%%%%%%%%%%%%%%%%%%%%%%%%%%%%%%%%%%%%%%%%%%%%%%%%%%%%%%%%%%
\end{document}
