%%%%%%%%%%%%%%%%%%%%%%%%%%%%%%%%%%%%%%%%%%%%%%%%%%%%%%%%%%%%%%%%%%%%%%%%
%%%  THIS TEX FILE IS TO GENERATE PDF FILE FOR 
%%% 
%%%  COPYRIGHT (C) JIMMY LIN, 2013, UT AUSTIN
%%%%%%%%%%%%%%%%%%%%%%%%%%%%%%%%%%%%%%%%%%%%%%%%%%%%%%%%%%%%%%%%%%%%%%%%
\documentclass[11pt,a4paper]{article}
%%%%%%%%%%%%%%%%%%%%%%%%%%%%%%%%%%%%%%%%%%%%%%%%%%%%%%%%%%%%%%%%%%%%%%%%
%%%  PACKAGES USED IN THIS TEX SOURCE FILE
%%%%%%%%%%%%%%%%%%%%%%%%%%%%%%%%%%%%%%%%%%%%%%%%%%%%%%%%%%%%%%%%%%%%%%%%
\usepackage{geometry,amsthm,amsmath,graphicx,fancyheadings}
\usepackage[colorlinks,
            linkcolor=blue,
            anchorcolor=red,
            citecolor=green
            ]{hyperref}
\usepackage{listings}
\usepackage{color}
\usepackage{/Users/JimmyLin/workspace/latexTemplate/UTA_CS/JS}
\usepackage{/Users/JimmyLin/workspace/latexTemplate/UTA_CS/JSASGN}
%%%%%%%%%%%%%%%%%%%%%%%%%%%%%%%%%%%%%%%%%%%%%%%%%%%%%%%%%%%%%%%%%%%%%%%%
\definecolor{mygreen}{rgb}{0,0.6,0}
\definecolor{mygray}{rgb}{0.5,0.5,0.5}
\definecolor{mymauve}{rgb}{0.58,0,0.82}
\lstset{ %
  backgroundcolor=\color{white},   % choose the background color; you must add \usepackage{color} or \usepackage{xcolor}
  basicstyle=\footnotesize,        % the size of the fonts that are used for the code
  breakatwhitespace=false,         % sets if automatic breaks should only happen at whitespace
  breaklines=true,                 % sets automatic line breaking
  captionpos=b,                    % sets the caption-position to bottom
  commentstyle=\color{mygreen},    % comment style
  deletekeywords={...},            % if you want to delete keywords from the given language
  escapeinside={\%*}{*)},          % if you want to add LaTeX within your code
  extendedchars=true,              % lets you use non-ASCII characters; for 8-bits encodings only, does not work with UTF-8
  frame=single,                    % adds a frame around the code
  keepspaces=true,                 % keeps spaces in text, useful for keeping indentation of code (possibly needs columns=flexible)
  keywordstyle=\color{blue},       % keyword style
  language=Octave,                 % the language of the code
  morekeywords={*,...},            % if you want to add more keywords to the set
  numbers=left,                    % where to put the line-numbers; possible values are (none, left, right)
  numbersep=5pt,                   % how far the line-numbers are from the code
  numberstyle=\tiny\color{mygray}, % the style that is used for the line-numbers
  rulecolor=\color{black},         % if not set, the frame-color may be changed on line-breaks within not-black text (e.g. comments (green here))
  showspaces=false,                % show spaces everywhere adding particular underscores; it overrides 'showstringspaces'
  showstringspaces=false,          % underline spaces within strings only
  showtabs=false,                  % show tabs within strings adding particular underscores
  stepnumber=2,                    % the step between two line-numbers. If it's 1, each line will be numbered
  stringstyle=\color{mymauve},     % string literal style
  tabsize=2,                       % sets default tabsize to 2 spaces
  title=\lstname                   % show the filename of files included with \lstinputlisting; also try caption instead of title
}
%%%%%%%%%%%%%%%%%%%%%%%%%%%%%%%%%%%%%%%%%%%%%%%%%%%%%%%%%%%%%%%%%%%%%%%%
%%% MACROS CONTAINING THE FILE INFORMATION
%%%%%%%%%%%%%%%%%%%%%%%%%%%%%%%%%%%%%%%%%%%%%%%%%%%%%%%%%%%%%%%%%%%%%%%%
\renewcommand{\COURSE}{CS371D Distributed System}
\renewcommand{\LECTURER}{Lorenzo Alvisi}
\renewcommand{\TUTOR}{Chao Xie}
\renewcommand{\TASK}{Problem Set 02}
\renewcommand{\RELEASEDATE}{March. 19 2014}
\renewcommand{\DUEDATE}{March. 26 2014}
\renewcommand{\TIMECONSUME}{20 hours}
%%%%%%%%%%%%%%%%%%%%%%%%%%%%%%%%%%%%%%%%%%%%%%%%%%%%%%%%%%%%%%%%%%%%%%%%
%%% DOCUMENTATION STARTS FROM HERE 
%%%%%%%%%%%%%%%%%%%%%%%%%%%%%%%%%%%%%%%%%%%%%%%%%%%%%%%%%%%%%%%%%%%%%%%%
\begin{document}
%%%%%%%%%%%%%%%%%%%%%%%%%%%%%%%%%%%%%%%%%%%%%%%%%%%%%%%%%%%%%%%%%%%%%%%%
%% TITLE PAGE
%%%%%%%%%%%%%%%%%%%%%%%%%%%%%%%%%%%%%%%%%%%%%%%%%%%%%%%%%%%%%%%%%%%%%%%%
\begin{titlepage}
    \maketitle
\end{titlepage}
%%%%%%%%%%%%%%%%%%%%%%%%%%%%%%%%%%%%%%%%%%%%%%%%%%%%%%%%%%%%%%%%%%%%%%%%
%% CONTENT PAGE: TABLEOFCONTENTS, LISTOFTABLES, LISTOFFIGURES
%%%%%%%%%%%%%%%%%%%%%%%%%%%%%%%%%%%%%%%%%%%%%%%%%%%%%%%%%%%%%%%%%%%%%%%%
\renewcommand{\contentsname}{Contents}
\begin{center} 
    \tableofcontents 
    %\listoftables 
    %\listoffigures
\end{center}
\newpage
%%%%%%%%%%%%%%%%%%%%%%%%%%%%%%%%%%%%%%%%%%%%%%%%%%%%%%%%%%%%%%%%%%%%%%%%
%%% GENERAL DOCUMENTATION BEGINS 
%%%%%%%%%%%%%%%%%%%%%%%%%%%%%%%%%%%%%%%%%%%%%%%%%%%%%%%%%%%%%%%%%%%%%%%%

\section{Problem 1: Early-Stopping Algorithm}
According to the class note, we first list out the property required by one
protocol if it is a correct t-tolerant early-stopping protocol as follows:

\noindent
\textbf{Validity}: If the sender is correct and broadcasts a message $m$, then
all correct processes eventually deliver $m$.

\noindent
\textbf{Integrity}: Every correct process delivers at most one message,
and if it delivers $m \neq SF$, then some process must have broadcast $m$.

\noindent
\textbf{Agreement}: If a correct process delivers a message $m$, then all correct
processes eventually deliver $m$.

\noindent
\textbf{Termination}: Every correct process eventually delivers some message.

\subsection{Under Crash Failures}
%{{{
\begin{proof}[Proof for Validity]
    A message $m$ sent from a correct sender to all processes at round 1 will be
    received by every correct process at the end of round 1. 
    % EXPLAIN
    This is true because of the validity of underlying $send$ and $receive$. 
    % CONCLUDE
    Hence, in this protocol, every correct process will deliver $m$ by the end of
    round 1. 
\end{proof}

\begin{proof}[Proof for Integrity]
    % TODO: not quite sure
The new protocol shows that each process delivers $at most$ one message and
then halts. The delivered message could either be $SF$ or $m$ depending upon
the sender action in round 1. For a process $p$ to deliver $m$ in a round $r
\neq 1$, it must have received the value from some other process $q$. In this
form, we can trace back the chain whose terminal point is the sender $s$ that
broadcast $m$. (Lemma 1 of the class note.)
\end{proof}
  
\begin{proof}[Proof for Agreement]
There can be no round where a process set value to $SF$ and another process
set value to $m$. This is proved as follows. Consider a round $k$ when process
$p$ decides $m$ and process $q$ decides $SF$. For process $p$ to decide $m$ it
either heard $m$ from the sender in round 1 or $m$ was relayed to it by some
other process. For $q$ to decide $SF$, it must have gone through two
consecutive rounds (say $i,j$) hearing only question mark $'?'$.
This is a contradiction as, if $p$ had heard the value $m$ in round $i$ it
would have broadcast $m$ in round $j$. Since all channels are reliable, $q$ is
will hear $m$ in round $j$ for sure. 
% CONCLUSION
Hence, we can conclude that no two processes can in the same round deliver different messages.  \\
\end{proof}

\begin{proof}[Proof for Termination]
    % FIRST BRANCH
    Based on the statement at line 6 and 8, if in any round a process receives
    a "non-$?$" value, then it delivers the value in that round.
    % SECOND BRANCH
    Based on the new statemnt at line 9 (actually it is line 10), if $?$ value
    is received, either it is in $f+1$ round or there are no new failure
    observed current round, $SF$ (sender fails) is delivered.
    % CONCLUDE
    Hence, we can conclude from the analysis of two cases above that
    Termination property of new protocol holds.
\end{proof}
%}}}

\subsection{Under Send-Omission Failures}
%{{{
We can disprove the correctness of the above protocol under existence of send
omission by . Postulate that a process $p$ exhibits send omission when it omits
to send a message $m$ to some or all of processes.  
Consider that in round 1, $s$ omits to send message $m$ to some processes,
say, $P$. Hence, we have $\forall p \in P, |faulty(p, 1)| = 1$. Then in Round 2, the sender now sends $m$ to a subset 
of the processes $P$ and their are no other faults. Those process receiving
$m$ will deliver $m$. But there are some processes, say $P_1$, still not get $m$
will set their $\forall p' \in P_1, |faulty(p, 2)| = 1$. For the set of processes not receiving $m$, no new faults have 
occurred, which satisfies $|faulty(p,k)| = |faulty(p,k-1)|$ (for here, $k=2$). Therefore, they
will decide that the sender is faulty, deliver $SF$ and halt.
    % DERIVE
    For now, we have some correct processes deliver $m$ and some
    correct deliver $SF$. 
    % CONCLUDE
    But it is apparent that this violates the property of agreement.
%}}}
\newpage

\section{Problem 2: Terminating Reliable Broadst}
\subsection{Algorithmic Description}
%{{{
    In order to solve TRB in $f+2$ rounds, we first specify the behavior of
    sender and receiver for round $1$ and then reference to the existing $f+1$
    consensus algorithm to decide (deliver) $m$. 

\begin{lstlisting}[language=java]
/*#########################*\
#  Sender s in first round  #
\*##########################*/
  send m to all other processes

/*#########################*\
#  Receiver in first round  #
\*#########################*/
  receive m from s (it can only be m)
  if m is received,
      then val = m 
  else val = SF .
\end{lstlisting}

\begin{lstlisting}[language=java]
// For this part, we directly reference the consensus algorithm in class note

// For the next f+1 rounds each process i executes the following 
Initially V = {v_i}  // value received from the first round
To execute propose(v_i):
    round k, 1<=k<=f+1
  send v_i to all
  for all j, 0<=j<=n-1, j != i  do
      receive value from process p_j
      V := V U value

// Process decide the consensus value as
To execute decide(x):
  if k = f+1: 
      decide V
\end{lstlisting}
%}}}

\newpage
\subsection{Correctness Check}
%{{{
\begin{proof}[Proof of Termination]
    It is obvious that no matter what sender broadcast at the first round, the
    new algorithm provided will send eventually deliver message $m$ or $SF$,
    in terms of the termination property of consensus algorithm.
\end{proof}
\begin{proof}[Proof of Validity]
% FOR ROUND ONE
Since we only postulate the existence of crash failure, a correct sender
broadcasting message $m$ will not be omitted. Hence, according to the
underlying $send$ and $receive$, the message $m$ broadcast by the sender at
 round $1$ will be obtained by all correct process. 
% FOR F+1 ROUND 
For round $2$ to round $f+1$, due to the validity condition of consensus
algorithm, all correct processes will eventually decide on $m$ and therefore delivering $m$.
\end{proof}
\begin{proof}[Proof of Agreement]
If a process deliver a message $m$, it must be value derived by the consensus
algorithm (consensus value). By the agreement 
property of the consensus algorithm, we can say that $m$ will be also
delivered by all correct processes. That is to say, a process can only deliver
a consensus value as all the other processes in the system do.
\end{proof}

\begin{proof}[Proof of Integrity]
Obviously, in the consensus algorithm, each process will eventually decide one
value.  Thus, no matter what value is initialized, we can say that every
correct process will consequently deliver $at most$ one message for the
resulted TRB algorithm.
%
And following the integrity property of the consensus algorithm, a message $m$
delivered by one process must be consensus message. And this consensus message
must be the one broadcast by the sender in round $1$ (the $send$ and
$receive$ are guaranteed).
For a process to deliver $m$ which is also the consensus message it has to be
proposed by some process in the system. This follows from the integrity
property of the consensus algorithm. By the algorithm given this is the $m$
that would have been broadcast by the sender in round $1$. The algorithm
maintains the channel $send$ and $receive$ property so message $m$ will be
received by all correct process at the end of round $1$. From there it will be
proposed and eventually delivered after $f+1$ rounds.
\end{proof}
%}}}
\newpage
\section{Problem 3: Uniform TRB}
\subsection{Round-based Algorithm}
%{{{


%}}}

\subsection{Unsolvable if $n \leq 2t$}
%{{{
\begin{proof}[Proof by Contradiction]
% assumption
Assume that there is a protocol that solves the uniform TRB under the
condition of faulty processes $t \geq n/2$ . 
% follow the hint
And then let us follow the hint to partition the processes in the system into two
disjoint sets $S_0$ and $S_1$, such that $S_0$ includes t 
processes (sender inclusive), and $S_1$ includes the other $n-t$ processes.
And we specify processes in $S_0$ be correct and all processes in set $S_1$ to be faulty.
% S_1
Note that by general omission failure, it can be send omission and receive
omission. That is, process in $S_1$ do neither receive messages (receive omission)
nor send messages (send omission) to processes in set $S_0$. 
% reasoning
Now we start the reasoning. When the sender in $S_0$ sends a message $m$, say
at round $r$, all
processes in $S_0$ receive the message and deliver it. However, processes in $S_1$
do not receive message because of receive omission. In the next round $r+1$, all 
processes will broadcast their received value to other processes.
Nevertheless, processes in $S_1$ send to each other $'?'$ since they have
receive nothing in round $r$. But Processes in $S_0$ send $m$ to all processes
since they are correct processes and thus receive value from last round $r$. 
% induce to end
By the same reasoning, processes in $S_1$ will not receive
a value and eventualy deliver $SF$ when the protocol ends.
But processes in $S_0$ deliver the first valid message $m$.
Hence, we get a situation where majority processes deliver $SF$ and minority
processes deliver initial message $m$, which violates uniform agreement.
% majority voting
And even we turn to majority voting to find the correct message, since the
majority processes are faulty (not communicable), we cannot eventually decide
the real value (recover) and reach uniform agreement.
Therefore there exists no algorithm to achieve uniform agreement if $2t \geq
n$.

\end{proof}
%}}}
%%%%%%%%%%%%%%%%%%%%%%%%%%%%%%%%%%%%%%%%%%%%%%%%%%%%%%%%%%%%%%%%%%%%%%%%
%%% General Documentation ends
%%%%%%%%%%%%%%%%%%%%%%%%%%%%%%%%%%%%%%%%%%%%%%%%%%%%%%%%%%%%%%%%%%%%%%%%
\end{document}
