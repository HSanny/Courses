%%%%%%%%%%%%%%%%%%%%%%%%%%%%%%%%%%%%%%%%%%%%%%%%%%%%%%%%%%%%%%%%%%%%%%%%
%%%  THIS TEX FILE IS TO GENERATE PDF FILE FOR 
%%% 
%%%  COPYRIGHT (C) JIMMY LIN, 2013, UT AUSTIN
%%%%%%%%%%%%%%%%%%%%%%%%%%%%%%%%%%%%%%%%%%%%%%%%%%%%%%%%%%%%%%%%%%%%%%%%
\documentclass[11pt,a4paper]{article}
%%%%%%%%%%%%%%%%%%%%%%%%%%%%%%%%%%%%%%%%%%%%%%%%%%%%%%%%%%%%%%%%%%%%%%%%
%%%  PACKAGES USED IN THIS TEX SOURCE FILE
%%%%%%%%%%%%%%%%%%%%%%%%%%%%%%%%%%%%%%%%%%%%%%%%%%%%%%%%%%%%%%%%%%%%%%%%
\usepackage{geometry,amsthm,amsmath,graphicx,fancyheadings}
\usepackage[colorlinks,
            linkcolor=blue,
            anchorcolor=red,
            citecolor=green
            ]{hyperref}
\usepackage{listings}
\usepackage[vlined,linesnumbered]{algorithm2e}
\usepackage{color}
\usepackage{/Users/JimmyLin/workspace/latexTemplate/UTA_CS/JS}
\usepackage{/Users/JimmyLin/workspace/latexTemplate/UTA_CS/JSASGN}
%%%%%%%%%%%%%%%%%%%%%%%%%%%%%%%%%%%%%%%%%%%%%%%%%%%%%%%%%%%%%%%%%%%%%%%%
\definecolor{mygreen}{rgb}{0,0.6,0}
\definecolor{mygray}{rgb}{0.5,0.5,0.5}
\definecolor{mymauve}{rgb}{0.58,0,0.82}
\lstset{ %
  backgroundcolor=\color{white},   % choose the background color; you must add \usepackage{color} or \usepackage{xcolor}
  basicstyle=\footnotesize,        % the size of the fonts that are used for the code
  breakatwhitespace=false,         % sets if automatic breaks should only happen at whitespace
  breaklines=true,                 % sets automatic line breaking
  captionpos=b,                    % sets the caption-position to bottom
  commentstyle=\color{mygreen},    % comment style
  deletekeywords={...},            % if you want to delete keywords from the given language
  escapeinside={\%*}{*)},          % if you want to add LaTeX within your code
  extendedchars=true,              % lets you use non-ASCII characters; for 8-bits encodings only, does not work with UTF-8
  frame=single,                    % adds a frame around the code
  keepspaces=true,                 % keeps spaces in text, useful for keeping indentation of code (possibly needs columns=flexible)
  keywordstyle=\color{blue},       % keyword style
  language=Octave,                 % the language of the code
  morekeywords={*,...},            % if you want to add more keywords to the set
  numbers=left,                    % where to put the line-numbers; possible values are (none, left, right)
  numbersep=5pt,                   % how far the line-numbers are from the code
  numberstyle=\tiny\color{mygray}, % the style that is used for the line-numbers
  rulecolor=\color{black},         % if not set, the frame-color may be changed on line-breaks within not-black text (e.g. comments (green here))
  showspaces=false,                % show spaces everywhere adding particular underscores; it overrides 'showstringspaces'
  showstringspaces=false,          % underline spaces within strings only
  showtabs=false,                  % show tabs within strings adding particular underscores
  stepnumber=2,                    % the step between two line-numbers. If it's 1, each line will be numbered
  stringstyle=\color{mymauve},     % string literal style
  tabsize=2,                       % sets default tabsize to 2 spaces
  title=\lstname                   % show the filename of files included with \lstinputlisting; also try caption instead of title
}
%%%%%%%%%%%%%%%%%%%%%%%%%%%%%%%%%%%%%%%%%%%%%%%%%%%%%%%%%%%%%%%%%%%%%%%%
%%% MACROS CONTAINING THE FILE INFORMATION
%%%%%%%%%%%%%%%%%%%%%%%%%%%%%%%%%%%%%%%%%%%%%%%%%%%%%%%%%%%%%%%%%%%%%%%%
\renewcommand{\COURSE}{CS371D Distributed System}
\renewcommand{\LECTURER}{Lorenzo Alvisi}
\renewcommand{\TUTOR}{Chao Xie}
\renewcommand{\TASK}{Problem Set 03}
\renewcommand{\RELEASEDATE}{April. 22 2014}
\renewcommand{\DUEDATE}{May. 02 2014}
\renewcommand{\TIMECONSUME}{20 hours}
%%%%%%%%%%%%%%%%%%%%%%%%%%%%%%%%%%%%%%%%%%%%%%%%%%%%%%%%%%%%%%%%%%%%%%%%
%%% DOCUMENTATION STARTS FROM HERE 
%%%%%%%%%%%%%%%%%%%%%%%%%%%%%%%%%%%%%%%%%%%%%%%%%%%%%%%%%%%%%%%%%%%%%%%%
\begin{document}
%%%%%%%%%%%%%%%%%%%%%%%%%%%%%%%%%%%%%%%%%%%%%%%%%%%%%%%%%%%%%%%%%%%%%%%%
%% TITLE PAGE
%%%%%%%%%%%%%%%%%%%%%%%%%%%%%%%%%%%%%%%%%%%%%%%%%%%%%%%%%%%%%%%%%%%%%%%%
\begin{titlepage}
    \maketitle
\end{titlepage}
%%%%%%%%%%%%%%%%%%%%%%%%%%%%%%%%%%%%%%%%%%%%%%%%%%%%%%%%%%%%%%%%%%%%%%%%
%% CONTENT PAGE: TABLEOFCONTENTS, LISTOFTABLES, LISTOFFIGURES
%%%%%%%%%%%%%%%%%%%%%%%%%%%%%%%%%%%%%%%%%%%%%%%%%%%%%%%%%%%%%%%%%%%%%%%%
\renewcommand{\contentsname}{Contents}
\begin{center} 
    \tableofcontents 
    %\listoftables 
    %\listoffigures
\end{center}
\newpage
%%%%%%%%%%%%%%%%%%%%%%%%%%%%%%%%%%%%%%%%%%%%%%%%%%%%%%%%%%%%%%%%%%%%%%%%
%%% GENERAL DOCUMENTATION BEGINS 
%%%%%%%%%%%%%%%%%%%%%%%%%%%%%%%%%%%%%%%%%%%%%%%%%%%%%%%%%%%%%%%%%%%%%%%%

\section{Problem 1: Leader Election Protocol}
\subsection{Pseudo-code Representation}
First of all, we thought of all algorithms that requires the comparison between
the identity of processes and found that the time complexity was at least
$O(n log_2(n))$. And we have to give up this approach.

We realize that this question actually involves in the distributed version of
trade-off between time complexity and space complexity. Since linear space
complexity is quite tricky, we start to think of protocols with high time
complexity and finally it pays off.

We develop a multi-round solution for leader election in synchronous
uni-directional ring network. This is because the communication is
synchronous. The designed protocol for leader election is as follows. 

Here, we refer the terminology "next" and "prev" to the process that it
connects to and that are connecting to it, respectively.

\begin{algorithm}[H]
 \KwIn{$n$ Processes with unique $id$}
 \KwIn{a ring-shape network with uni-directional and synchronous communication}
 \KwResult{a system with new leader confirmed by all $n$ processes}
\end{algorithm}

\noindent \\
For each replica, 

\begin{itemize}
    \item{All processes are initially sleeping. }
    \item{At every round, each process $i$ determines whether it is going to wake up by itself.}
        \begin{itemize}
        \item{Process $i$ is not allowed to wake up if it receives a leader
                request from prev}
        \item{If waken up, send a leader request with its id $i$ to next.}
        \item{If not wake up, do nothing.}
        \end{itemize}
    \item{In the following round, when each process $i$ receives leader
            request from process $j$}
        \begin{itemize}
            \item{If $j < i$, reject the leader request with id $j$ since the
                    spread of leader request with id $i$ is slower than that of
                    process id $j$. And process $j$ propose its leader request
                to its next.}
            \item{If $j > i$, retain leader request with id $j$ for $2^j$
                    rounds. Note that we do this exponential retainment to
                    distinguish leader request in different message(signal) frequency.}
            \item{If $j = i$, the leader request returns back to the proposer
                    and hence process $i$ becomes leader of this distributed
                    system.}
        \end{itemize}

\end{itemize}


\subsection{Correctness Proof}
In the case that all processes wakes up in the same round, (actually this is
the unresolvable scenario for most protocols) all processes send its leader
request to their respective "next" process. And according to the well-designed
mechanism, the spread of leader request for different processes lies in
different frequency period. Specifically, the leader request with lower id
value will traverse through the ring network faster than those requests with
higher id value. That is, in this case, the process with lowest $id$ will
have its leader request message going through the ring network in a fastest
rate. And hence it defeats all other competing processes, and finally become
the only new leader. 

Here we justify the fact that the messages used in the above case is linear on
the total nubmer of processes. For leader request with lowest of id $i$
traversing the whole ring network, the number of delivery made to the leader
request with id $j$ is $\frac{n}{2^{j-i}}$. By summing over $j-i$ from $1$ to
$\infty$, it can be derived that the total number of messages employed are no
more than 

$$ \sum_{j-i=1}^{+\infty}\frac{n}{2^{j-i}} \leq
\frac{1-(\frac{1}{2})^{+\infty}}{1-\frac{1}{2}} \cdot n = 2n$$

For other cases, it is obvious to see that the messages required is less than
the scenario presented above and the lowest awakened process will finally
become the unique leader.

\newpage

\section{Problem 2: Solvability of S failure detector}

\begin{algorithm}
$V_p$ := ($\perp$, $\dots$, $\perp$, vp, $\perp$, $\dots$, $\perp$) \{$p$'s
    estimate of the proposed values\} \\

$\bigtriangleup _p$ := ($\perp$, $\dots$, $\perp$, vp, $\perp$, $\dots$, $\perp$)  \\
\{phase 1\} \{asynchronous rounds $r_p$, $1 \leq r_p \leq n – 1$\} \\
\For { $r_p$ := 1 to n-1 } 
    { send ($r_p$, $\bigtriangleup _p$, $p$) to all \\
      wait until [$\forall q$: received ($r_p$, $\bigtriangleup_q$, $q$) or
        $q \in D_p$]  \\
    Op := $V_p$ \\
    $V_p$ := $V_p$ $\oplus$ ($\oplus q$ received $\bigtriangleup_q$) \\
    $\bigtriangleup _p$ := $V_p$ $\star$ $O_p$ {value is only echoed first time
        it is seen}  \\
    }
\{phase 2\} \\
send ($r_p$, $V_p$, $p$) to all \\
wait until [$\forall q$: received ($r_p$, $V_q$, $q$) or $q \in D_p$] \\
$V_p$ := $\otimes q$ received $V_q$ \{computes the "intersection", including
    $V_p$\} \\
\{phase 3\} \\
decide on leftmost non-$\perp$ coordinate of $V_p$ \\
\end{algorithm}

\subsection{Proof for Validity}
{\bf Uniform validity}: If a process decides a value $v$, then some process
proposed $v$.

In the phase three, process $p$ make its
decision based on the first non-$\perp$ element of the $V_p$, which summarizes
all cumulative responses from other processes over $n-1$ rounds (in phase 1) and those
responses (in phase 2). Hence, the decision $v$ made by process $p$
must have been proposed by some process. And it can be concluded that the
solution provided by S failure detector satisfies the validity property.

\subsection{Proof for Termination}
{\bf Termination}: Every correct process eventually decides some value. 

For those communication in phase 1 and phase 2, if the process $q$ is faulty,
it will be detected by the failure detector. Otherwise, the response from
process $q$ can be guaranteed (eventually). Hence, there will be no blocking on the "wait"
command. Therefore, it can be fully guaranteed that once the process starts
executing this protocol, it will eventually reach the terminal state.

\subsection{Proof for Agreement}
{\bf Agreement}: No two correct processes decide differently.

In phase 2, every process sends messages to ask for response to its decided
value and proceed to phase 3 with the value resulted by intersecting its own
decision in phase 1 and others' response in phase 2. In this way, the derived
$V_p$ would have the same value on its left-most non-$\perp$ position. Hence,
no two correct would decide differently.

\newpage

\section{Problem 3: Ben-Or Lite}
After removing the line 12, the algorithm is of the following type.

\begin{algorithm}[H]
    $a$ := input bit; $r$ := 1 \\
    repeat forever \\
    \{phase 1\} \\
    send ($a_p$,$r$) to all \\
    Let $A$ be the multiset of the first $n-f$ $a$-values with timestamp $r$ received \\
    if ($\exists v \in \{0, 1\}: \forall a \in A$) then $b_p$ := $v$ \\
    else $b_p =\ \perp$ \\
    \{phase 2\} \\
    send ($b_p$,$r$) to all \\
    Let $B$ be the multiset of the first $n-f$ $b$-values with timestamp $r$ received \\
    if ($\exists v \in \{0, 1\}: \forall b \in B$) then  decide($v$) \\
    . \\
    else $a_p := \$ \ \{\$$ is chosen uniformly at random to be 0 or 1\} \\
    $r$ := $r$ + 1
    \caption{Ben-Or Lite Algorithm}
\end{algorithm}

\vspace{0.3cm}

It can be eaisly observed that processes could fail to learn from the values
decided by other processes without the line of 12. In this way, the agreement
property will be violated. Hence, we can say that the consensus cannot be
satisfied by Ben-Or Lite Algorithm.

\section{Problem 4: Disprove consensus solution}
Disproof: The Validity property will be violated. 
The decision value may probably be different from the proposed value.

The problem lies in the fact that if two processes propose $b$ and $b'$
sequentially. The final decided version could be overlapped version by $b$ and
$b'$, that is, $b_0,b'_1,b'_2,b_3,...,b_n$ and $b'_0,b_1,b_2,b'_3,...,b'_n$. 

%%%%%%%%%%%%%%%%%%%%%%%%%%%%%%%%%%%%%%%%%%%%%%%%%%%%%%%%%%%%%%%%%%%%%%%%
%%% General Documentation ends
%%%%%%%%%%%%%%%%%%%%%%%%%%%%%%%%%%%%%%%%%%%%%%%%%%%%%%%%%%%%%%%%%%%%%%%%
\end{document}
